\chapter{QDP Command summary}
\label{qdpcommands}

QDP commands must be inserted at the beginning of a QDP file,
as these commands tell QDP how to read in the data.
Any command not recognized by QDP is passed to PLT\@.
QDP separates command lines from data lines
based on the first non-blank character in the line.
If this character is {\tt +}\, , {\tt -}\, , {\tt .}\, , or a digit,
then the entire line will be read as data.

\section*{READ Serr}
\begin{verbatim}
READ Serr [vlist]
\end{verbatim}
Tell QDP/PLT which vectors have symmetric errors.
The command \ {\tt READ~Serr~1~3~5} \ will cause
vectors 1, 3, and 5 to be read with symmetric errors,
and vectors 2 and 4 to be read without.
Only one \ {\tt READ~Serr} \ command should appear in a QDP file.
 
\medskip
\noindent {\em Example:}
\begin{verbatim}
READ Serr 1 3 5
 1. .1     2.    3. .3    4.   5. .5
\end{verbatim}
This would be read as 5 vectors:
$1.0\pm 0.1$, $2.0$ (no error), $3.0\pm 0.3$,
$4.0$ (no error), and $5.0\pm 0.5$.
Without the \ {\tt READ~Serr} \ command,
it would be read as 8 vectors.
 
\section*{READ Terr}
\begin{verbatim}
READ Terr [vlist]
\end{verbatim}
Tell QDP/PLT which vectors have two-sided errors.
It takes three columns to specify a vector with two-sided errors.
The first column is the central value,
the second column (which must be positive) specifies the upper bound,
and the third column (which must be negative or zero) specifies the lower bound.
 
\medskip
\noindent {\em Example:}
\begin{verbatim}
READ Serr 1
READ Terr 2
 1. .1    2. +.1 -.2
\end{verbatim}
This would be read as $1.0\pm 0.1$ and $2.0^{+0.1}_{-0.2}\,$\@.
Note: In fitting, non-positive errors are ignored;
thus the first error of two-sided errors should be positive.
