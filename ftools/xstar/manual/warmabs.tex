\chapter{WARMABS}
\label{sec:warmabs}

An alternative method for fitting XSTAR results to observed spectra
within xspec is to use the xspec `analytic' model warmabs.
This actually includes several separate models: warmabs, photemis, 
hotabs, hotemis, windabs and multabs.  These are described below.

This model allows the use of xspec models for warm absorbers and photoionized emitters,
and for coronal equilibrium absorbers and emitters
without requiring construction of mtables or etables.  The advantages 
of this procedure include:

1) Circumventing the intrinsic approximations associated with 
use of tables for absorption with variable abundances treated 
as multiplicative parameters.  (see the `Important Notes on Mtables' section of the 
xstar2xspec chapter of the xstar manual).   Warmabs/photemis/windabs/multabs  
calculate spectra using stored level populations which are then scaled using 
element abundances specified during the xspec session before the spectra are 
calculated.  Therefore the approximations associated with the use of 
tables are avoided.  (However, see the section below on the current 
limitations of these models).

2) Circumventing the intrinsic clumsiness of the use of tables.

3) Ability to use arbitrary spectral resolution, not limited 
by the internal xstar spectral resolution

3.5) Allows the use of turbulent broadening as a fitting parameter in 
xspec.

4) This model employs the most recent updates to xstar and the database, 
version 2.2. 


\section{Obtaining warmabs}

As of this writing, the warmabs package is not included as 
part of the standard xstar distribution, either within heasoft 
or as part of standalone xstar.  Instead, it must be downloaded 
and installed separately from the ftp site:
ftp://legacy.gsfc.nasa.gov/software/plasma\_codes/xstar/warmabs22.tar.gz

The contents of the tarfile include:

atdbwarmabs.fits  -- xstar atomic database binary fits file.
This must reside in directory pointed to by the WARMABS\_DATA 
environment variable.  This file has the same format  as the atdb.fits file 
used by xstar, but is kept distinct.  {\bf This is important} because the 
structure of the populations files (described below) depend on the 
version of the atomic data used in the run which created them.  Thus 
warmabs must use the appropriate version of this data file, the one 
included in its distribution.  The version of atdb.fits used by the headas or 
xstar installation may not be identical.

coheatwarmabs.dat -- compton heating/cooling data file.
This must reside in directory pointed to by WARMABS\_DATA 
environment variable.  This file is identical to the file 
coheat.dat which is part of xstar.

fphotems.f -- source code for warmabs, photemis, windabs, 
multabs, hotemis, and hotabs

lmodel\_warmabs.dat -- local model definition file needed by xspec.  
When using xspec11, must be renamed to lmodel.dat or catted 
onto existing lmodel.dat

pops.fits.nxx  binary fits files containing pre-calculated level populations 
for use with the warmabs or  photemis model.   Currently the user chooses 
which of these to use using the WARMABS\_POP environment variable, eg., by 
doing 'setenv WARMABS\_POP pops.fits.n4' (note that the full path is not used
here, rather the path relative to the WARMABS\_DATA path).

popshot.fi -- binary fits file containing pre-calculated level populations for the hotabs and
hotemis models. This file also must reside in directory pointed to by 
WARMABS\_DATA environment variable. This file is not intended for modification by the user

instructions.txt --  text version of the xspec11 manual page describing the  
setup of analytic models

xautosav.xcm -- sample xspec script file showing use of model.

README     -- this file

\section{Installation}

The procedure for setting up and using this model is as described in the 
xspec manual:


0) You need to have the heasoft package installed on your 
machine, but it must be built from source.  Local models 
cannot be installed from a binary installation.

1) untar this directory somewhere in your user area

2) setup your headas environment (eg. 'setenv HEADAS /path/to/architecture',
and 'source \$HEADAS/headas-init.csh')


3) point the WARMABS\_DATA enviroment variable to the directory 
where the warmabs data files atdbwarmabs.fits, coheatwarmabs.dat and 
pops.fits are kept (eg, 'setenv WARMABS\_DATA \$PWD').  

4) In addition the WARMABS\_POP environment variable must be set to point to the 
name of the populations file (relative to the WARMABS\_DATA location) to be used
by the  warmabs, windabs, multabs  and photemis models.  For example, 
the populations for a density 10$^4$ gas illuminated by a $\Gamma$=2 power law 
are in pops.fits.n04, so to use these do 'setenv WARMABS\_POP pops.fits.n04'.

5) Start up xspec, and in response to the prompt type 
'initpackage xstarmod lmodel\_warmabs.dat <path-to-current-directory>',
where <path-to-current-directory> is the full path of the current directory 
and can be the dot ('.').
Then, after the build is complete type 
'lmod xstarmod <path-to-current-directory>'
In subsequent  sessions you don't neet to do the initpackage step again, just the lmod.

6) Instructions for xspec11:

a) Set the environment variable LMODDIR to the local directory,
where you have untarred the package, eg. 
'setenv LMODDIR \$PWD'

b) make sure this directory is  pointed to by your shared object library 
environment variable, as described in the xspec11 manual, eg.
'setenv LD\_LIBRARY\_PATH "${LMODDIR}:${LD\_LIBRARY\_PATH}"'

c) rename the lmodel\_warmabs.dat file to lmodel.dat, or cat it onto an 
existing lmodel.dat

d) cd into  \$LHEASOFT/../spectral/xspec/src/local\_mod

e) type hmake 

f) cd to the directory where you want to work and start xspec 11.


\section{warmabs}

Inside of xspec, the model can be invoked by typing 
'mo warmabs*pow' or variations on that.
The input parameters include: absorber column, parameterized as 
log(N/10$^{22}$ cm$^{-2}$), log(ionization parameter) element abundances 
relative to solar, turbulent broadening in km/s, and redshift.

\section{photemis}

The photoionized emitter can be invoked by typing 'mo photemis'.  This model is 
the `thermal' (i.e. recombination and collisional excitation) emission which 
comes from the same plasma used in warmabs.  Note that this does NOT include the 
resonant scattered emission associated with the warmabs line emission; the windabs
model can be used, with a trick, to model this emission if desired (see below).

The model supplies to xspec the emissivity of the gas, in units of 
erg cm$^{-3}$ s$^{-1}$, times a factor 10$^{10}$.  So the physical meaning of the 
normalization, $\kappa$, is

\begin{equation}
\kappa=\frac{{\rm EM}}{4\pi D^2} \times 10^{-10}
\end{equation}

\noindent where EM is the emission measure of the gas 
in the source (at the ionization parameter used in the fit) and D is the distance to 
the source.  This same relation applies to the results from the hotabs model.

\section{windabs}

Windabs uses the 'sei' method of 
Lamers et al. 1987 Ap. J. 314 726 to take the line center optical depth 
calculated by the usual warmabs routines and spread them in wavelength 
space in order to model the scattering 
in an outflowing wind.  Bound-free absorption is treated 
the same as in warmabs.  For lines, the optical depths from warmabs
are input to the sei routines as the value of the 
paramter ttot (optical depth normalization).  
The input velocity vturb is used as the terminal velocity of the wind.
 Other sei parameters are assumed to be fixed:  gamma=1 (run of optical 
depth with velocity), apha=1 (velocity law), and eps=0 
(thermalization parameter).
Windabs has the same input parameters as warmabs, 
except that it also has a covering fraction parameter, $C$ 
(not part of the standard sei formulation),
which should be in the range 0-2.  
The physical meaning of $C$ is that, when 
0$\leq C \leq$1, the emission component is reduced by a factor of $C$.
A feature added on 09/04/2007: when covering fraction $C\geq$1, 
the absorption component is reduced by a factor $2-C$.  So at $C=2$, the model 
produces pure scattered emission.  Values of $C\geq$2 have no physical meaning.

\section{multabs}

Multabs tries to account for line broadening by absorption by multiple discrete components 
rather than by turbulence or bulk flow.  This model is essentially identical 
to warmabs in that it uses the warm absorber spectrum generated by warmabs, but 
initially assuming that the lines are broadened only by thermal gas motions. 
It then replicates these lines a fixed number of times and spreads the components 
over a given velocity width.  The input parameters include the same parameters 
as  warmabs:  ionization parameter, column, and abundances.  These control the properties
of the individual absorbing components, in the same way as for warmabs.
In addition, the user specifies the velocity spread of the components, still called 
vturb, and the covering fraction, cfrac.  This covering fraction 
can be interpreted as a covering fraction in velocity space.  
The number of discrete components is given by cfrac*vturb/vtherm,
where vturb is input by the user and vtherm is the thermal line width which 
is determined by the equilibrium temperature (calculated by xstar).  The 
optical depth of each component is divided by the number of components, so that the 
total optical depth summed over the components is independent of their number.
The number of components cannot be less than 1. If it is 1 then the component will 
be placed at the redshifted energy of the line.  If it is greater than 1, then the 
components will be spaced uniformly  in velocity from -vturb to +vturb relative 
to the rest energy of the line.  There is no restriction on the value of cfrac, 
so it is possible to set cfrac to some large number and thereby fill a uniform 
trough in velocity with the line.

\section{hotabs}

Hotabs and hotemis are the coronal analogs of warmabs and photemis.  In this 
case the free parameter determining the ionization is the log of the temperature
in units of 10$^4$ K, i.e. logt4=0 corresponds to 10$^4$ K, logt4=3 corresponds to 10$^7$K.


The normalization is the same as for warmabs. 
The model supplies to xspec the emissivity of the gas, in units of 
erg cm$^{-3}$ s$^{-1}$, times a factor 10$^{10}$.  So the physical meaning of the 
normalization, $\kappa$, is

\begin{equation}
\kappa=\frac{{\rm EM}}{4\pi D^2} \times 10^{-10}
\end{equation}

\noindent where EM is the emission measure of the gas 
in the source (at the temperature used in the fit) and D is the distance to 
the source.  


\section{Creating Your Own pops.fits Files}

The procedure for doing this is as follows:

a) {\bf You must use consistent versions of warmabs and xstar version 2.2.}
Xstar is available either as standalone or as 
part of the heasoft distribution.  That this is done correctly can be confirmed by comparing the 
opening banner of the warmabs run to the opening banner of an xstar run.  The warmabs banner shows 
both the warmabs version and the associated xstar version.  This is because 
warmabs uses many xstar routines and the xstar atomic data file, and the pops.fits file
structure depends on the atomic data file.   We attempt to maintain 
consistency between the warmabs version in the tarfile at 
ftp://legacy.gsfc.nasa.gov/software/plasma\_codes/xstar/warmabs22.tar.gz
with the current release xstar version associated with the headas installation.  
We also attempt to maintain 
consistency between the warmabs develop version in the tarfile at 
ftp://legacy.gsfc.nasa.gov/software/plasma\_codes/xstar/warmabs22dev.tar.gz
with the current develop xstar version available at
ftp://legacy.gsfc.nasa.gov/software/plasma\_codes/xstar/xstar22src.tar.gz.
It is generally recommended to use the latter pair of codes when generating your own pops.fits files.
Versions of xstar earlier than 2.2 cannot be used to generate population files 
for use by warmabs and photemis.

b) Using xstar version 2.2, run a variation of the constant density sphere
described in the manual chapter 2, with low density and low luminosity so that the sphere 
is optically thin. warmabs and photemis treat the ionization parameter as the free parameter
describing level populations, so it is desirable that the model run in xstar
be optically thin, and that it span the range of ionization parameter of interest.
An example of the xstar command run from the command line is:

xstar cfrac=0. temperature=10000. pressure=0.03 density=10000. spectrum='pow'  trad=-1. rlrad38=1.e-10  column=1.e+17   rlogxi=5.  lcpres=0 habund=1. heabund=1.  cabund=1. nabund=1. oabund=1. neabund=1. mgabund=1. siabund=1.  sabund=1.  arabund=1. caabund=1. feabund=1. niabund=0. modelname="otfs" niter=99  npass=1 critf=1.e-7 nsteps=10 xeemin=0.04 emult=0.1 taumax=50. lprint=1 lwrite=1

and an example of the xstar.par file which could be used instead is:

\begin{verbatim}
cfrac,r,a,1,0.,1.,"covering fraction"
temperature,r,a,10000,0.,1.e4,"temperature (/10**4K)"
lcpres,i,a,0,0,1,"constant pressure switch (1=yes, 0=no)"
pressure,r,a,0.03,0.,1.,"pressure (dyne/cm**2)"
density,r,a,1.e+4,0.,1.e18,"density (cm**-3)"
spectrum,s,a,"pow",,,"spectrum type?"
spectrum_file,s,a,"spct.dat",,,"spectrum file?"
spectun,i,a,0,0,1,"spectrum units? (0=energy, 1=photons)"
trad,r,a,-1,,,"radiation temperature or alpha?"
rlrad38,r,a,1.00E-015,0.,1.e10,"luminosity (/10**38 erg/s)"
column,r,a,1.00E+016,0.,1.e25,"column density (cm**-2)"
rlogxi,r,a,5,-10.,+10.,"log(ionization parameter) (erg cm/s)"
habund,r,a,1,0.,100.,"hydrogen abundance"
heabund,r,a,1,0.,100.,"helium abundance"
liabund,r,h,0,0.,100.,"lithium abundance"
beabund,r,h,0,0.,100.,"beryllium abundance"
babund,r,h,0,0.,100.,"boron abundance"
cabund,r,a,1,0.,100.,"carbon abundance"
nabund,r,a,1,0.,100.,"nitrogen abundance"
oabund,r,a,1,0.,100.,"oxygen abundance"
fabund,r,a,1,0.,100.,"fluorine abundance"
neabund,r,a,1,0.,100.,"neon abundance"
naabund,r,a,1,0.,100.,"sodium abundance"
mgabund,r,a,1,0.,100.,"magnesium abundance"
alabund,r,a,1,0.,100.,"aluminum abundance"
siabund,r,a,1,0.,100.,"silicon abundance"
pabund,r,a,1,0.,100.,"phosphorus abundance"
sabund,r,a,1,0.,100.,"sulfur abundance"
clabund,r,a,1,0.,100.,"chlorine abundance"
arabund,r,a,1,0.,100.,"argon abundance"
kabund,r,a,1,0.,100.,"potassium abundance"
caabund,r,a,1,0.,100.,"calcium abundance"
scabund,r,a,1,0.,100.,"scandium abundance"
tiabund,r,a,1,0.,100.,"titanium abundance"
vabund,r,a,1,0.,100.,"vanadium abundance"
crabund,r,a,1,0.,100.,"chromium abundance"
mnabund,r,a,1,0.,100.,"manganese abundance"
feabund,r,a,1,0.,100.,"iron abundance"
coabund,r,a,1,0.,100.,"cobalt abundance"
niabund,r,a,1,0.,100.,"nickel abundance"
cuabund,r,a,1,0.,100.,"copper abundance"
znabund,r,a,1,0.,100.,"zinc abundance"
modelname,s,a,"otfs",,,"model name"
nsteps,i,h,3,1,1000,"number of steps"
niter,i,h,0,,,"number of iterations"
lwrite,i,h,0,0,1,"write switch (1=yes, 0=no)"
lprint,i,h,0,0,2,"print switch (1=yes, 0=no)"
lstep,i,h,0,,,"step size choice switch"
emult,r,h,0.5,1.e-6,1.e+6,"Courant multiplier"
taumax,r,h,5.,1.,10000.,"tau max for courant step"
xeemin,r,h,0.1,1.e-6,0.5,"minimum electron fraction"
critf,r,h,1.e-7,1.e-24,0.1,"critical ion abundance"
vturbi,r,h,1.,0.,30000.,"turbulent velocity (km/s)"
radexp,r,h,0.,-3.,3.,"density distribution power law index"
ncn2,i,h,999,999,99999,"number of continuum bins"
loopcontrol,i,h,0,0,30000,"loop control (0=standalone)"
npass,i,h,1,1,10000,"number of passes"
mode,s,h,"ql",,,"mode"
\end{verbatim}

It is important to have the write switch set to 1 in order to 
generate the file containing the level populations at each step.

In doing this, you may want to change the shape of the ionizing spectrum.
In these examples it is a power law with $\gamma$=2, which is the 
same as what is used in the distributed warmabs/photemis package.
The maximum ionization parameter is controlled by the 
value of the variable rlogxi, in this case it is 5, corresponding
to $\xi$=1.e5.  The minimum ionization parameter
is set by the column density of the model; this input file 
will terminate when log($\xi$) falls below -2.  The number of 
intermediate steps is controlled by the variable nsteps; in this case
nsteps=6 corresponds to a uniform spacing of 0.13 in log($\xi$).
If the final model has more than 200 spatial zones, warmabs and photemis 
will discard any populations for spatial zones beyond 200.

c) After the run is complete, the populations are in the file xo0x\_detail.fits
where x is the number of the final pass (usually 1).
This must be moved to the directory containing the xstar data files, i.e. the 
directory pointed to by the WARMABS\_DATA environment variable, and then 
pointed to by the WARMABS\_POP environment variable.  Then warmabs 
and photemis will read populations from this file.
The file lmodel.dat, as it is distributed, contains limits on the 
ionization parameter which may not be appropriate to your file pops.fits, so 
you may want to change this.

\section{Common block `ewout'}


A feature added September 2007 is output of the strongest lines, 
sorted by element and ion into a common block called 'ewout'
This feature is only available for the warmabs, photemis, hotemis, hotabs  
models (not windabs, or multabs). The contents of the common block are:

lmodtyp: identifies which model most recently put its output into the 
         common block.  lmodtyp=1,2,3,4, where 1=hotabs,
         2=hotemis, 3=warmabs, 4=photemis 

newout:  number of lines in the list.  This is zeroed after each call to warmabs, 
         photemis, hotabs, etc.

lnewo:   array conatining line indexes.  These should correspod to the line indexes 
         in the ascii line lists on the xstar web page.

kdewo:   character array containing the name of the ion

kdewol:  character array containing the name of the lower level

kdewou:  character array containing the name of the upper level

aijewo:  array containing A values for the lines

flinewo: array containing f values for the lines

ggloewo: array containing statistical weights for the lower levels

ggupewo: array containing statistical weights for the upper levels

elewo:   array containing the line wavelengths

tau0ewo: array containing the line center depths

tau02ewo:array containing the line depths at the energy bin nearest to line center

ewout:   array containing line equivalent widths in eV, negative values correspond to emission

elout:   array containing line luminosities in xstar units (erg/s/10\^38)


The details of how to get at the contents of the common block are up to the user.
Currently xspec does not have a mechanism to do this, but it is straightforward to write 
a small fortran code to call the models with suitable parameter values and print the 
common block from there.  The calling sequence for 
an analytic model is described in the xspec manual.
It is important to point out that the common block is 
overwritten at each call to one of the models, so it should be emptied by the calling 
program after each call to one of the models.


An example is as follows:

\begin{verbatim}
      program fphottst
c
      implicit none
c
      real ear(0:20000),photar(20000),photer(10000),param(30)      
      integer ne,mm,ifl
      real emin,emax,dele
c
      ne=10000
      emin=0.4
      emax=7.2
      dele=(emax/emin)**(1./float(ne-1))
      ear(0)=emin
      do mm=1,ne
        ear(mm)=ear(mm-1)*dele
        enddo
      write (6,*)ear(1),ear(ne),ear(ne/2)
      param(1)=2.
      param(2)=-4.
      param(13)=100.
      param(12)=0.
      param(3)=1.
      param(4)=1.
      param(5)=1.
      param(6)=1.
      param(7)=1.
      param(8)=1.
      param(9)=1.
      param(10)=1.
      param(11)=1.
      param(3)=0.
      call fhotabs(EAR,NE,PARAM,IFL,PHOTAR,PHOTER)
c
      call commonprint
c
      write (6,*)'after fwarmabs'
      do mm=1,ne
        write (6,*)ear(mm),photar(mm)/ear(mm)
        enddo
c
      stop
      end
      subroutine commonprint
c
      implicit none  !jg
c
      parameter (nnnl=200000)
c
      common /ewout/newout,lnewo(nnnl),kdewo(8,nnnl),
     $  kdewol(20,nnnl),kdewou(20,nnnl),aijewo(nnnl),flinewo(nnnl),
     $  ggloewo(nnnl),ggupewo(nnnl),
     $  elewo(nnnl),tau0ewo(nnnl),tau02ewo(nnnl),ewout(nnnl),
     $  elout(nnnl),lmodtyp
c
      real aijewo,flinewo,ggloewo,ggupewo,elewo,tau0ewo,tau02ewo,
     $  ewout,elout
      integer lnewo,newout,lmodtyp
      character*1 kdewo,kdewol,kdewou
      integer kk,mm   !jg
c
       if (lmodtyp.eq.1) write (6,*)'after hotabs',newout
       if (lmodtyp.eq.2) write (6,*)'after hotemis',newout
       if (lmodtyp.eq.3) write (6,*)'after warmabs',newout
       if (lmodtyp.eq.4) write (6,*)'after photemis',newout
       write (16,*)'index, ion, wave(A), tau0, tau0grid, ew (eV),',
     $ 'lum, lev\_low, lev\_up, a\_ij, f\_ij, g\_lo, g\_up'
        do kk=1,newout
         write (16,9955)kk,lnewo(kk),(kdewo(mm,kk),mm=1,8),
     $     elewo(kk),tau0ewo(kk),tau02ewo(kk),ewout(kk),
     $     elout(kk),
     $     (kdewol(mm,kk),mm=1,20),(kdewou(mm,kk),mm=1,20),
     $     aijewo(kk),flinewo(kk),ggloewo(kk),ggupewo(kk)
         enddo
9955   format (1x,2i8,1x,8a1,5(1pe11.3),1x,2(20a1,1x),4(1pe11.3))
c
      return
      end
\end{verbatim}


New in version 2.02:  inclusion of continua, both in emission and absorption.  
Equivalent widths are not calculated, and quantities analogous to the 
transition probability and oscillator strength are not output.  Also, the 
upper level, which may not be the ground level of the adjacent ion, is 
not identified.  

New in version 2.03: Fix to error in normalization of voigt profile.

New in version 2.04: Rational and uniform level labels for all levels.
These should now be unambiguous.  A description is contained in the xstar 
manual.  Also, the interface with the commonprint common block has been 
updated.  There is now a (string) variable whiche denotes whether a 
transition is a line or rrc/edge.  Upper levels for rrc/edges are denoted
'continuum' for the ground state of the next ion, or by the appropriate 
level string when the upper level is not the ground state.

New in version 2.06:  Consistency with xstar version 221bn.  Includes 
up to date r-matrix atomic data for Ni and Al.  Also includes model
'scatemis', which allows calculation of emission models for resonant-excitation
dominated plasmas (optically thin).

New in version 2.07:  Contents of the common block are output to fits files.
A new file is created with each call to warmabs or photemis, with names like
'warmabsxxxx.fits', where xxxx is a sequential number, mod 9999. In order 
to implement this feature, the fphotems.f file must be edited:  calls to 
the routine 'fitsprint' must be uncommented.

New in version 2.07b:  Fixed minor errors in lmodel.dat.  Shifted O I and O II 
L alpha to match observations.

New in version 2.09:  Made consistent with xstar v221bn15.  New N VI collisional 
excitation data.

New in version 2.10:  Made consistent with xstar v221bn17.  Changes to naming of output fits file suggested by John Houck:  if the WARMABS\_OUTFILE environment variable is set, then this name is used for the output fits file.  Also checks to make sure that the pops.fits file was created with the same atomic database as the current one, and exits if not.

New in version 2.11:  O I absorption cross sections from Gorczyca et al. (2013). Fine structure for H-like ions for Z$\>$20.
Multiple errors fixed 10/08/2013.

New in version 2.12:  Ne I absorption cross sections from Gorczyca et al. (2013). Fine structure for all H-like ions.

New in version 2.13:  Mg I-III absorption cross sections from Gorczyca et al. (2013). 
New input parameters for warmabs, photemis, hotabs, hotemis:  
write\_outfile is a switch controlling output of line depths/luminosities to an ascii fits file:  
0=no fits files produced, 
1=fits file containing lines/edges is produced, 
2=fits file containing ion column densities is produced, 
3=both types of fits files 
are produced.
outfile\_idx is an integer index.  If the environment WARMABS\_OUTPUT is not set, 
then the output is written to a fits file names 'warmabsxxxx.fits', where
xxxx is the value of this variable.  If WARMABS\_OUTPUT is set, then its 
value is used as the name of the output file.
Use of xstar v221bn18 routines:  update to fundamental constants, adding 
thermal and turbulent velocities in quadrature consistently.

New in version 2.14:  extrapolation of all valence shell cross sections beyond 
tabulated values.  This is a temporary fix to the problem of apparent 
'negative edges' which appear at large column density in absorption spectra.
Needed are extensions to the tabulated cross sections to higher energies.

New in version 2.15:  Undid extrapolation of all valence shell cross 
sections beyond tabulated values because this led to spurious 
cross sections in some cases (He0 ground --> He+ 2p).  Instead manually
inserted extrapolated He0 ground --> He+ ground cross section 
into atomic data.  Implemented cosmic abundances gotten from xspec 
internal tables.  These can be changed using the 'abund' command, 
as described in the xspec manual.

New in version 2.17:  when the 'write\_outfile' parameter value is 
nonzero an additional file is produces, called warmabs\_columns.  This 
file contains the column densities of all ions from the most 
recent model call.

New in version 2.18:  Fixed bug which affected output to the fits output 
files for photemis.

New in version 2.19:  Fixed additional bug which affected output to the 
fits output files for photemis.  This resulted in unphysical small wavelengths
tabulated in the file for various lines, and also caused the wavelengths 
to change between successive calls.  Also changed the use of the 
write\_outfile variable such that the values are as follows:  0=no fits files
produced, 1=fits file containing lines/edges is produced, 2=fits file 
containing ion column densities is produced, 3=both types of fits files 
are produced.  Also increased the critical luminosity needed for photemis
to add line emision to the spectrum, in order to speed execution.

New in version 2.20:  Fixed error in Si XIV energy levels which was introduced
when putting fine structure.

New in version 2.21:  Added fine structure of He-like ions of C-Ni.

New in version 2.22:  Added list of level indeces and ion index to the fits 
table output.  These indeces are unlikely to change over time as new atomic 
data is added, so this should provide a robust way to keep track of 
lines and rrcs.

New in version 2.23:  Removed duplications of atomic data: rrcs in 
Ni IX -- XV and of lines in Na X and F VIII.

New in version 2.24:  Changes to xstar code corresponding to xstar v2.2.1bn24.

New in version 2.25:  Fix to error which led to incorrect emission in the Fe UTA.

New in version 2.26:  Compatibility with xstar version 2.3.  Extend photoionization 
extrapolation from 20 keV to 200 keV.

New in version 2.27:  Compatibility with xstar version 2.31. 

\section{Limitations}

This package is still being tested.  Some embarrassing bugs have already been found, 
but more may still lurk.  Please contact me with any reports or questions.

2) It is not blindingly fast.  On a 1GHz machine, the first time
'mo warmabs ..' is typed, the initial setup requires $\sim$30 seconds.
After that, each time an abundance or ionization parameter is changed, 
it requires 5-10 seconds to recalculate the model.  Presumably 
on a faster machine this will be reduced.

3) It calculates the spectrum 'on the fly', appropriate to the energy grid
and parameters the user specifies.  But it does not calculate the 
ionization balance self-consistently.  It uses a saved file of 
level populations calculated for a grid of optically thin models calculated
with a $\Gamma=2$ power law ionizing spectrum.  So this will not be self-consistent 
if your source has a very different ionizing spectrum.  Also it implicitly 
assumes that the absorber has uniform ionization even if you specify 
a large column, which is not self-consistent.

4) Photemis does not take into account scattering, only true emission.

