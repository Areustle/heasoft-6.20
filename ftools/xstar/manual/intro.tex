
\chapter{Introduction}
\label{sec:introduction}
\pagenumbering{arabic}

\section{What is XSTAR?}

XSTAR is a command-driven computer program for calculating
the physical conditions and emission spectra of photoionized gases.
It may be applied in a wide variety of astrophysical
contexts.  Stripped to essentials, its job may be described simply:
A spherical gas shell surrounding a central source of ionizing radiation
absorbs some of this radiation and reradiates it in other portions of
the spectrum; XSTAR computes the effects on the gas of absorbing this
energy, and the spectrum of reradiated light.  In many cases other sources
(or sinks) of heat may exist, for example, mechanical compression or
expansion, or cosmic ray scattering.  XSTAR permits consideration
of these effects as well.  The user supplies the shape and strength
of the incident continuum, the elemental abundances in the gas, its
density or pressure, and its thickness; the code 
returns the ionization balance and temperature,
opacity, and emitted line and continuum fluxes.
The solution divides into several distinct parts: transfer of
the incident radiation into the cloud; calculation of the temperature,
ionization, and atomic level populations at each point in the cloud; and transfer of the emitted radiation
out of the cloud.
XSTAR v2 is written in standard fortran77, and has been tested on a 
variety of unix platforms

\section{Scope of This Document}

The new user will need to first read Chapter~\ref{sec:installation} on how to obtain a copy
of XSTAR, and is then advised to read Chapter~\ref{sec:walkthrough} which gives 
the flavor of an XSTAR session, and provide sufficient information to get started.
The chapters \ref{sec:overview} and \ref{sec:xstarinput}, give an overview of
the code structure and describe the input commands.
The output is described in Chapter~\ref{sec:output}.  More experienced users may wish to read 
Appendix~\ref{sec:physics}, which
discusses the physical assumptions made in XSTAR, or 
Appendix~\ref{sec:internals}, on the code structure.


We emphasize here and throughout this document that XSTAR version 2 represents 
an almost complete rewrite of version 1.  This affects not only the 
internal operation of the code and the numerical results, but also the user interface.
These changes were motivated primarily by the desire for a more streamlined operation, and for 
flexibility incorporating future and current improvements in atomic data.  
In addition, we have tried to learn from experience gained from version 1 by 
eliminating options which were seldom used in favor of clearer and more 
straightforward specification of input parameters, and we have replaced 
the code which parsed the input commands and which did not function as 
required in some situations.  In order to do so we have 
borrowed heavily from the well developed input and output code, and installation 
scripts, which have been developed in this laboratory for the FTOOLS 
software package.  This includes the use of FITS files for some output,
As a consequence, it will be necessary for even experienced users of version 1 to 
become familiar with the new interface.



\section{Acknowledgments}

We would like to thank many colleagues for their suggestions, 
bug reports, and (occasionally)
source code. The initial development of XSTAR was promoted by
Prof. Dick McCray at the University of Colorado.  Contributions to the code
and atomic database have come from John Raymond, Barry Smith, Ian Stevens and 
Yuan-Kuen Ko.  Much of the impetus for work on versions 1 and 2 came from Julian Krolik.
The work for version 2 could not have been carried out without 
programming support and advice  from Tom Bridgman, James Peachey, Bryan Irby, and Bill Pence.
A great deal of crucial work on the atomic data was done by Manuel Bautista, and 
also by Patrick Palmeri, Claudio Mendoza, Javier Garcia, Mike Witthoeft and Ming-Feng Gu.  
The production of this manual and the circulation of the code has been funded 
by NASA through the Astrophysical Data Program, Grant NAG 5-1732.

\section{Useful addresses}
\begin{tabular}{lll}
	Timothy Kallman, Code 662  & Internet: & tim@xstar.gsfc.nasa.gov  \\
	NASA GSFC & Telephone: & (301)-286-3680  \\
	Greenbelt MD 20771, USA  & FAX: &  (301)-286-1684  \\
\end{tabular}
