\chapter{The Atomic Database}
\label{sec:atomicdatabase}


\section{General Description}

The database system used by XSTAR version 2 attempts to 
separate, as much as possible, the numerical quantities which 
determine the various atomic rates from the fortran code 
which actually performs the calculation.  The goal is 
make the atomic data modular, so that new data can be 
adopted or tested without requiring extensive modifications to the 
code.  The way this is done is to separate the data from the code itself,
and store the data in a database which is designed specifically for 
use by XSTAR.  The database is divided into 
`records', each of which corresponds to a given physical 
process affecting a given level or pair of levels. An example is the 
radiative decay of hydrogen from the 2p to the 1s level.
Each record contains numerical constants needed to calculate the 
rate for the process, in this example simply the Einstein A value 
for the transition, together with various other associated quantities.
Chief among these are two integers which describe
the how the constants are to be used.  The first integer 
is denoted the `data type', and describes the fitting formula to be used 
in order to calculate a rate from the constants.  The second 
integer is the `rate type', which describes how XSTAR uses the 
rates calculated.  The list of data types is already quite long 
and is expected to grow and change as new data is adopted into the database,
but not all data types are used by the current database.
In order to interpret the various data types, XSTAR contains 
one central data calculating subroutine, denoted ucalc.f, 
which branches to various segments of code (and calls to specialized 
subroutines) which are tailored to each data type.  
ucalc.f returns the rates in a standard form for use by the other 
XSTAR subroutines.  It is expected that ucalc.f will require additions
in order to handle new data types as they are adopted.  
The list of rate types is not intended to grow, since such changes 
could require changes to the rest of the XSTAR code structure.


The XSTAR database system can be divided into 3 parts:

First, and most important, is the ascii file containing all the data.  That is,
this file contains all the numerical data and labels required for 
calculation of all atomic rates and resultant quantities.
This includes all level excitation energies, statistical weights and 
spectroscopic names, all element names and abundances, all ion names,
and of course all photoionization cross sections, collision rates, 
recombination rates, fluorescence yeilds, and line wavelengths.
This file is separated into records, corresponding crudely to lines of text, 
although many records extend over more than one line.  
Each record consists of a header, followed by the data.  
The header currently consists of 6 integers:  the data type, the rate type, 
a continuation flag (currently unused), the number of reals in the 
record, the number of integers in the record, and the numbers of 
characters in the record.  Then follows the real data, the integer data, and 
the character data.  The various fields within the record 
are separated by one or more spaces.  The record is terminated with a $\%$, and the 
entire database is terminated by a single line containing $\%\%\%\%$.
Each record can currently contain up to 2000 of any of the types of constants:
real, integer, or character.  In the XSTAR source tree 
this file is named atdat.text and currently is approximately 10MB in size.

In order to facilitate rapid reading of this file by XSTAR, it is 
converted into two binary fits tables.  The first one contains the header data
for each record, the second contains a concatenation of all the non-header data.
They are named, respectively, aptrs.fits and atdat.fits.  Operation of 
XSTAR requires that the environment variable LHEA$\_$DATA be set to the 
directory containing these files.  In the standard distribution these data files
are kept in the refdata directory in the ftools area, and the appropriate 
value of the environment variable LHEA$\_$DATA is set by the script which 
initializes FTOOLS, lhea-init.csh (c shell) or lhea-init.sh (bourne shell).

The third part of the database is the subroutine ucalc.f.  This routine, when 
passed the contents of a record, returns the result of the rate 
calculation for the corresponding process.  ucalc therefore contains all 
of the various arithmetic expressions corresponding to rates for various 
physical processes.  ucalc returns generally 4 real rates and two integers.
The rates are: rate, inverse rate, heating rate, and cooling rate.  
The integers are indeces of the levels involved, lower and upper.
Not all data types return all 4 rates.

The list of rate types currently included in ucalc.f are as follows: 


\begin{description}

 \item[1] ground state ionization      

 \item[3] bound-bound collision        

 \item[4] bound-bound radiative        

 \item[5] bound-free collision (level) 

 \item[6] total recombination          

 \item[7] bound-free radiative (level) 

 \item[8] total recombination, forces norm 

 \item[9] 2 photon decay                

 \item[11] element data                 

 \item[12] ion data                     

 \item[13] level data                   

 \item[14] bound-bound radiative superlevel-spectroscopic level     

 \item[40] bound-bound collisional superlevel-spectroscopic level     

 \item[41] non-radiative Auger transition

 \item[42] Inner shell photoabsorption followed by autoionization

\end{description}


The list of data types currently included in ucalc.f are as follows: 
Those denoted xstar1 are not in use in the standard distribution of XSTAR 
version 2, but are maintained in order to facilitate comparison with 
the results of XSTAR version 1.

In what follows, T=temperature in $10^4$K; r1,r2,...=real numbers in record;
i1,i2,...=integers in record; c1,c2,...=characters in record


\begin{description}

\item[1) Radiative recombination, Aldrovandi and Pequignot formula:]
 $$rate=r1/T^{r2}$$

\item[2) Charge exchange with $H^0$, from Kingdon and Ferland:]

\begin{verbatim}
      rate=aax*expo(log(t)*bbx)*(1.+ccx*expo(ddx*t))*(1.e-9)
      ans1=rate*xh0
\end{verbatim}

where xh0 is the number density of neutral hydrogen.

\item[3) Autoionization correction to DR, formula from Hamilton, Sarazin 
and Chevalier:(xstar1)]
 $$rate=r1*e^{r2/kT}/T^{1/2}$$

\item[4) Line data:]
 r1=line wavelength (A)
; r2=f value
; i1=lower level index
; i2=upper level index

\item[5) 2 photon transition, collisional excitation.(xstar1)]
 r1=line wavelength (A)
; r2=f value
; r3=lower level g
; r4=upper level g
; i1=lower level index
; i2=upper level index
; r5=collision strength at kT= $\varepsilon_{line}$

\item[6) Level data:]
 Contains level information for ions. Energies are in eV.      
 Every line  contains:
 reals r1=E$_Z$=1(eV); r2=(2J+1); r3=n$_{effective}$; r4=Ion. Potential;
i1=n; i2=(2S+1); i3=L; i4=Z; i5=$\#$lev; i6=$\#$ion)
; c1-c$\#$=configuration.

\item[7) Dielectronic recombination, aldrovandi and Pequignot formula:]
 $$rate=r1*10^{-6}*e^{-r2/T}*(1+r3*e^{-r4/T})/T^{3/2}$$

\item[8) Dielectronic recombination, Arnaud and Raymond formula:]
 $$rate=r1*e^{-r2/kT} + r3*(T^{-r4-r5*ln(T)})$$

\item[9) Charge exchange with $He^0$, formula from Kingdon and Ferland:]


\item[10) Charge exchange with $H^+$, formula from Kingdon and Ferland:]:]                            
\begin{verbatim}
      rate=aax*t**bbx*(1.+ccx*expo(ddx*t))
     $             *expo(-eex/t)*(1.e-9)
      ans1=rate*xh1
\end{verbatim}
 

\item[11) 2 photon decay data:(xstar1)]
; r1=line wavelength (A)
; r2=f-value
; i1=lower level index
; i2=upper level index

\item[12) Photoionization cross section, broken power law:]
{\bf (xstar1)} 
 $$\sigma=r1*(\varepsilon/r5)^{r2}$$
 i1=level index

\item[13) element data:]
 r1=abundance, r2=mass, i1=int(z), i2=index, c1-c30=name

\item[14) Ion data:]
 r1=ionization threshold, c1-c8=name

\item[15) Photoionization cross section, Barfield, Koontz \& Huebner scaled from 
neutrals:]{\bf (xstar1)}
 $$cross section=10^{\Sigma_{n=1}^{n=12}C_n (ln(\varepsilon/\varepsilon_0))^{n-1}-18}$$
 for $\varepsilon \leq$ r3
; where $\varepsilon$=photon energy in eV,$\varepsilon_0$=r1-$\Delta$,
 $\Delta$=r2, and $C_1$=r4, $C_2$=r5.

\item[17) Line collision data:  hydrogenic isosequence, rates from Cota:(xstar1)]
 r1=line wavelength (A)
; r2=f value
; r3=lower level g
; r4=upper level g
; i1=lower level index
; i2=upper level index
; r5=collision strength at kT= $\varepsilon_{line}$

\item[18) Radiative Recombination rates for H-like levels, rates from Cota:(xstar1)]
 $$rate=10^{r1+r2*(log_{10}(T)+4.-r3)^2}$$

\item[19) Photoionization cross section from HULLAC:]
(not used).
 

\item[20) Same as 10, but used for total rate]                        

\item[ 21)  ]
 

\item[22) Dielectronic Recombination, Storey low temperature: (xstar1)]
 $$rate=10^{-12}*(r1/T+r2+T*(r3+T*r4))*T^{3/2}*e^{-r5/T}$$
 

\item[23) Photoionization cross section, Clark et al. formula: (xstar1)]
 

\item[25) Collisional Ionization data from Raymond and Smith:(xstar1)]
 r1=e
; r2=a
; r3=b
; r4=c
; r5=d
 $$ch = 1./chi$$
 $$fchi = 0.3*ch*(a+b*(1.+ch)+(c-(a+b*(2.+ch))*ch)*alpha+d*beta*ch)$$
 $$rate = 2.2e-6*sqrt(chir)*fchi*expo(-1./chir)/(e*sqrt(e))$$

\item[26) Collisional Ionization data from Cota, H-like levels:(xstar1)]
 idest1=idat(1)
; gglo=rlev(2,idest1)
; edelt=abs(rlev(1,idest1)-rlev(1,nlev))
; ans1=(4.1416e-9)*rdat(1)*t**rdat(2)*exp(-edelt/ekt)
            /gglo
; ans1=ans1*xnx


\item[27) photoionization: hydrogenic]
 

\item[28) line data collisional: Mendoza; Raymond and Smith: (xstar1) ]
 

\item[29) collisional ionization data: scaled hydrogenic: (xstar1) ]
 

\item[30) Radiative Recombination, hydrogenic, total, Gould and Thakur formula:]
 $$rate=2*(2.105 \times 10^{-22})*vth*y*\phi$$
 where:
\begin{verbatim}
        zeff=r1
        beta=zeff*zeff/(6.34*t6)
        yy=beta
        vth=(3.10782e+7)*sqrt(t)
c       fudge factor makes the 2 expressions join smoothly
        ypow=min(1.,(0.06376)/yy/yy)
        fudge=1.*(1.-ypow)+(1./1.5)*ypow
        phi1=(1.735+alog(yy)+1./6./yy)*fudge/2.
        phi2=yy*(-1.202*alog(yy)-0.298)
        phi=phi1
        if (yy.lt.0.2525) phi=phi2
\end{verbatim}

\item[31) line data including statistical weights for upper and lower(xstar1)]
 

\item[32) Collisional ionization, Cota, ground level.(xstar1)]
 idest1=idat(1)=1
; gglo=rlev(2,idest1)
; edelt=abs(rlev(1,idest1)-rlev(1,nlev))
; ans1=(4.1416e-9)*rdat(1)*t**rdat(2)*exp(-edelt/ekt)
             /gglo
; ans1=ans1*xnx

\item[33) line data collisional: hullac (not used)          ]
 

\item[34) line data radiative: Mendoza and from Raymond and Smith (xstar1) ]
 

\item[35) photoionization: table (from Barfield, Koontz and Huebner 1972):
(xstar1) ]
 

\item[36) photoionization, excited levels, hydrogenic(no l): (xstar1) ]

i1=n, i4=level; i5=ion
 
\item[49) opacity project  pi x-sections                                  ]
 Photoionization cross section from TOPbase averaged over resonances. 
 Photon energies are in Ry with respect to the subshell ionization threshold 
and cross sections are in Mb.
Just like 53, except for energy scale..
 Every line  contains:
 reals (2*np; x(i),y(i),i=1,np)
; integers (8; N, L, 2*J, Z, $\#$lev+, $\#$nion+, $\#$nlev, $\#$nion)
; characters (0)
; ($\#$ion $\&$ $\#$nlev correspond to the initial state and $\#$ion+ $\&$ $\#$lev+
 correspond to the state to which that ionizes.) 


\item[50) line rad. rates from OP and IP]
 Transition probabilities file.
 Every line  contains:
 reals (3; Wavelength (A), gf-val., A-val. (cm-1))
; integers (4; lower level, upper level, Z, $\#$ion)
; characters (0)

\item[51) iron project and chianti line collision rates    ]
 Burgess $\&$ Tully fit to collision strengths as taken from CHIANTI.
 Each fit entry includes the C-fitting parameter and 5 reduced 
 collision strengths values for X=0, .25, .5, .75, 1.
; reals (7; Delta E, C fitting param., 5 Y-reduced values)
; integers (5; transition type, lower level, upper level, Z, $\#$ion)
; characters (0)

\item[52)  same as 59 but rate type 7   ]


\item[53) opacity project  pi x-sections                                  ]
 Photoionization cross section from TOPbase averaged over resonances. 
 Photon energies are in Ry with respect to the first ionization threshold 
and cross sections are in Mb.
 Every line  contains:
 reals (2*np; x(i),y(i),i=1,np)
; integers (8; N, L, 2*J, Z, $\#$lev+, $\#$nion+, $\#$nlev, $\#$nion)
; characters (0)
; ($\#$ion $\&$ $\#$nlev correspond to the initial state and $\#$ion+ $\&$ $\#$lev+
 correspond to the state to which that ionizes.) 

\item[54) Transition probabilities to be computed from quantum defect or as hydrogenic.  ]
 Transition probabilities most be included as hydrogenic.
 reals (1; 0.0E+0)
; integers (4; lower level, upper level, Z, $\#$ion)
; characters (0)

\item[55)  hydrogenic pi xsections, bautista format: ]

i1=lower level; i2=ion;



\item[56) Tabulated Upsilons for HeI from Sawey $\&$ Berrington (1993).]
 Every line contains:
 reals (2n; n log10(temp), n gammas)          
; integers (4; lower level, upper level, Z, $\#$ion)
; characters (0)

\item[57) Effective ion charge for each level to be used in collisional ionization rates]
(same as 65)
 Every line contains:
 reals (1, Zeff)
; integers (6; N, L, 2*J, Z, $\#$lev, $\#$ion)
; characters (0)

\item[58) Bautista cascade rates]
(not used)

\item[59) verner pi xc                                     ]

 Verner photoionization cross sections, after 
 D. A. Verner $\&$ D. G. Yakovlev, 1995, A$\&$AS, 109, 125
 r1-r6:  fitting parameters
; i1=nuclear z
; i2=number of electrons
; i3=subshell
; i4=verner fitting parameter, orbital quantum number of subshell
; i5=final ion stage-initial ion stage
; i6=final level number
; i7=ion

\item[60) calloway h-like coll. strength                   ]

 Coefficients for analytic fits to Upsilons
 for H-like ions according to review by Callaway (1994;
 ADNDT,
 57,9)
 Data lines contain the following information:
 reals (coefficients)
; integers (4; lower level, upper level, Z, $\#$ion)
; characters (5)

\item[61) Collision strengths from impact parameter approximation] 
(not used)

Every line contains: reals (0); integers (5: dummy, lower level, 
upper level, Z, $\#$ion); characters (0).

\item[62) calloway h-like coll. strength                   ]
(same as 60)
 Coefficients for analytic fits to Upsilons
 for H-like ions according to review by Callaway (1994;
 ADNDT,
 57,9)
 Data lines contain the following information:
 reals (coefficients)
; integers (4; lower level, upper level, Z, $\#$ion)
; characters (5)

\item[63) h-like cij,  (hlike ion)                 ]
 Transition probabilities to be computed from quantum defect or as 
 hydrogenic.
 reals (1; 0.0E+0)
; integers (4; lower level, upper level, Z, $\#$ion)
; characters (0)

\item[64) hydrogenic pi xsections, bautista format:]

i3=z; i1=n; i2=l;

\item[65) effective charge to be used in collisional ionization (h-like
ions)]
 Effective ion charge for each level to be used in collisional ionization
 rates.
 Every line contains:
 reals (1, Zeff)
; integers (5; N, L, Z, $\#$lev, $\#$ion)
; characters (0)

\item[66) Kato \& Nakazaki fit to collision strengths for He-like ions] 

Like type 69 but in fine structure.

 Every line contains:
 reals (6; fit coefficients)
; integers (4; lower level, upper level, Z, $\#$ion)
; characters (0)


\item[67) Effective collision strengths for He-like ions from Keenan, McCann, and Kingston]
{\bf (1987)} 
 Every line contains:
 reals (n; fit coefficients)
; integers (4; lower level, upper level, Z, $\#$ion)
; characters (0)

\item[68) Fit to effective collision strengths for He-like ions by Zhang $\&$ Sampson. ]
 Every line contains:
 reals (3; fit coefficients)
; integers (4; lower level, upper level, Z, $\#$ion)
; characters (0)

\item[69) Kato $\&$ Nakazaki (1996) fit to collision strengths for He-like ions.]
 Every line contains:
 reals (6; fit coefficients)
; integers (4; lower level, upper level, Z, $\#$ion)
; characters (0)

\item[70) Coefficients for recomb. and phot x-section of superlevels.]
 Every line  contains:
 reals ($\#$; (den(i),i=1,nd),(Te(i),i=1,nt),
        (log10(recomb. rates(i,j),i=1,nt,j=1,nd)
        (ener(i), pi x-secs(i), i=1,nx)
; integers (11; nd, nt, nx, N, L, 2*S+1, Z, $\#$lev+, $\#$nion+, $\#$nlev, $\#$nion)
; characters (0)

\item[71) Radiative transition rates from superlevels to spectroscopic levels]
 The data is for a grid of Ne and  
 Te. 
 Every line  contains:
 reals ($\#$; Ne(i),i=1,nd),(Te(i),i=1,nt),
        (rad. rates (ne,kt),kt=1,nt,ne=1,nd), Wavelength (\AA)) 
; integers (6; nd, nt, lower level, upper level, Z, $\#$ion)
; characters (0)

\item[72) Autoionization rates (in s$^{-1}$) for satellite levels.]
 Every line contains:
 reals (3; auto. rate, energy in eV above the ionization limit, 
            statistical weight)
; integers (6; (2S+1), L, $\#$level, continuum level numb., z, ion)
; characters (10; level configuration)

\item[73) Fit to effective collision strengths from Sampson et al. for satellite levels] 
{\bf of He-like ions.}
 Every line contains:
 reals (7; fit coefficients)
; integers (4; lower level, upper level, Z, $\#$ion)
; characters (0)

\item[74) Delta functions to add to phot. x-sections to match ADF DR recomb. rates.]
 Every line  contains:
; reals (2n+1; ionization limit (eV), (energy over g.s.(i),i=1,n),
        (amplitude in cm$^2$(i),i=1,n)
; integers (8; N, L, (2S+1), Z, $\#$lev+, $\#$nion+, $\#$nlev, $\#$nion)
; characters (0)

\item[75) Autoionization data for Fe XXiV satellites  :]

 Every line contains:
 reals (same as 72) (3; auto. rate, energy in eV above the ionization limit, 
            statistical weight)
; integers: lower level, upper ion, upper level, ion
; characters (0)

\item[76) 2 photon decay :]
 
Just like data type 50.
Every line  contains:
 reals (3; Wavelength (A), gf-val., A-val. (cm-1))
; integers (4; lower level, upper level, Z, $\#$ion)
; characters (0)

\item[77) Collision transition rates from superlevels to spectroscopic levels] 

Every line contains: reals ($\#$: (Ne(i),i=1,nd),(Te(i),i=1,nt),
coll.rates(ne,kt),ne=1,nt,kt=1,nd), Wavelength (\AA)); integers (6: nd,
nt, lower level, upper level, Z, $\#$ion); characters (0).

\item[78) Level data used for Auger and inner shell fluorescence 
calculation:]  Same as type 6:
 
Every line  contains:
 reals r1=E$_Z$=1(eV); r2=(2J+1); r3=n$_{effective}$; r4=Ion. Potential;
i1=n; i2=(2S+1); i3=L; i4=Z; i5=$\#$lev; i6=$\#$ion)
; c1-c$\#$=configuration.

Different data type used in order 
to merge with non-Auger levels when assembling database.  These data 
come from the compilation of Kaastra and Mewe (1993).  They are 
gradually being replaced by more accurate level-to-level data.
As of v.21l and later, this has been done for iron and for oxygen.
In version 2.1kn4 and earlier, this data was used for all elements.

\item[79)  Line data used for Auger and inner shell fluorescence 
calculation:]  Same as type 4, but different data type used in order 
to merge with non-Auger levels when assembling database.
 
 r1=line wavelength (A)
; r2=f value
; i1=lower level index
; i2=upper level index



\item[80) Collisional ionization rates gnd of Fe and Ni :]

not used

\item[81) Bhatia Fe XIX collision strengths  :] 

Every line  contains:
 reals r1=$\Upsilon$; ;
i1=lower level; i2=upper level; i3=$\#$ion);

Energy separation is obtained from level data.

\item[82) Fe UTA rad rates  :]         

(from Gu et al. 2006)

Reals: r1=wavelength (A); r2=; r3=gf; r4=A$^{radiative}_{ij}$; 
r5=A$^{Auger}_{ij}$; r6=A$^{total}_{ij}$;
i1=lower level; i2=upper level;

\item[83) Fe UTA level data  :]                          

(from Gu et al. 2006)

 Same as type 6:
 
Every line  contains:
 reals r1=E$_Z$=1(eV); r2=(2J+1); r3=n$_{effective}$; r4=Ion. Potential;
i1=n; i2=(2S+1); i3=L; i4=Z; i5=$\#$lev; i6=$\#$ion)
; c1-c$\#$=configuration.


\item[84) Iron K Pi xsections, spectator Auger binned    :]

No longer used

\item[85) Iron K Pi xsections, spectator Auger summed    :]

Calculates photoionization cross section due to summation of 
resonances near inner shell edges ala Palmeri et al., 
2002 Ap.J.Lett.577, 119.

Every line  contains:
 reals r2=E$_{Threshold}$(Ry); r3=f parameter; r4=$\gamma$; r5=scaling factor;
i1=lower level; i2=ion;



\item[86) Iron K Auger data :]

(from Palmer et al. 2003A$\&$A...410..359P, Mendoza et al  2004A$\&$A...414..377M;
Palmeri et al., 2003A$\&$A...403.1175P; Garcia et al.,  2005ApJS..158...68G)

reals: r2=A$_{ij}$ (s$^{-1}$); 
integers:
i1=$\#$ final level (relative to final ion); i4=$\#$final ion
i2=$\#$ initial level; i5=$\#$initial ion


\item[88) Iron inner shell resonance excitation  :]
Photoexcitation to autoionizing levels

Format is like types 49:
 Photon energies are in Ry with respect to the first ionization threshold 
and cross sections are in Mb.
 Every line  contains:
 reals (2*np; x(i),y(i),i=1,np)
; integers (8; N, L, 2*J, Z, $\#$lev+, $\#$nion+, $\#$nlev, $\#$nion)
; characters (0)
; ($\#$ion $\&$ $\#$nlev correspond to the initial state and $\#$ion+ $\&$ $\#$lev+
 correspond to the state to which that ionizes.) 


\end{description}

\section{Utility Programs}

The program which translates the ascii database file into the binary 
fits format used by XSTAR is called bintran.f, and is included 
with the XSTAR source distribution.  Compilation of this program is 
straightforward, although it requires links to the cfitsio libraries.
Execution simply requires the redirection of the input.


\section{Level Labels}

New in version 2.21bh is the replacement of all level strings by a uniform 
system developed for the the uadb database.  The following is reproduced from 
the uadb manual and describes the labeling system.


While level strings from any coupling scheme can be stored and retrieved from
uaDB, currently it only supports searching for $LS$-coupled level strings.
In order to guarantee uniqueness, level strings entered into the database must
conform to the rules outlined in this appendix.

All states must have a configuration.
Term-averaged or level-resolved states must also include a term string and
level-resolved states must specify $J$.
The rules for each part follow.


\subsection{Configuration strings}

Configurations are stored in the database using an unambiguous notation which
should be familiar to most users.
A configuration consists of a space-delimited list of sub-shells in standard
order each having the form, $nlm$, where $nl$ is the sub-shell (standard order: 1s, 
2s, 2p, 3s, ...) and $m$ is the occupation number.
Note that the shorthand notation of omitting $m$ when unity is not used, e.g. 
2s1 not 2s.
Configuration strings obey the rules:
\begin{itemize}
\item all closed sub-shells starting with 1s and ending just prior to the first
      open (or last) sub-shell are not part of the configuration string,
\item the first open sub-shell is always displayed even if it is empty ($m=0$),
      and
\item all empty sub-shells beyond the first open sub-shell are not 
      displayed.
\end{itemize} 
Some examples:
\begin{itemize}
\item 1s$^2$ 2s$^2$ 2p$^3$ becomes 2p3,
\item 1s$^2$ 2s$^1$ 2p$^4$ becomes 2s1 2p4,
\item 1s$^2$ 2s$^0$ 2p$^5$ becomes 2s0 2p5, and
\item 1s$^1$ 2s$^2$ 2p$^4$ becomes 1s1 2s2 2p4.
\end{itemize}

Using a list of occupation numbers as the configuration label was considered
and ultimately rejected due to the impracticality of storing Rydberg levels.
Consider the configuration, 1s 200p; whereas only 13 characters are
needed to store this configuration in the form described above, nearly 
40\,000 characters are required if using a list of occupation numbers.

To get the number of electrons of a configuration takes two steps; first you
need to calculate the number of electrons in the core and then add up the
occupation numbers of the visible sub-shells.
To get the number of electrons in the core, $n_{core}$, take the principal
quantum number, $n$, and the orbital angular momentum, $l$ of the first 
{\bf open} sub-shell and apply the following expression:
\begin{equation}
n_{core} = \frac{1}{3} n (n-1) (2n-1) + 2l^2 .
\end{equation}
For a configuration of 4p5 5s2 5p1 we have $n=4$ and $l=1$.
The above expression yields $n_{core} = 30$ and the total occupation of the
visible sub-shells is 8 so this configuration has 38 electrons.


\subsection{Term strings}

The format for the term should be familiar to most users. 
It starts with an integer representing $2S+1$ followed by the spectroscopic
letter representing the total orbital angular momentum, $L$.
An example is {\tt 2P} where $S=1/2$ and $L=1$.


\subsection{Level strings}

To specify the total angular momentum, $J$, of a level-resolved state, you 
append the term string defined above with an underscore and the $J$ value.
If $J$ is a half-integer then you must use fractional notation.
Examples of the term and level strings include: {\tt 2P\_1/2} and {\tt 1S\_0}.


