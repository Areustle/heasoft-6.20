\chapter{Sample Results}

\section{From ``Photoionization and High Density Plasmas'', 
T. Kallman and M. Bautista, June 2000, submitted to Ap. J.}


\section{ Sample Results: Low Density }

Although many of the results of XSTAR v.2 calculations are similar to those 
described in KM82, we present as background some results of 
simple models which illustrate the behavior of photoionized 
gas and which disply some of the adopted atomic rates and cross sections.
All of the results presented in this and the next section are for 
optically thin models; we defer a discussion of radiation transfer 
effects to a later paper.

\subsection{Atomic Rates and Cross Sections}

We begin by displaying some of the atomic rates 
which are notable due to their departure from previous work, or to 
their effects on the model results.
Figure 1 shows the ground state photoionization cross sections we adopt.
Each panel contains the cross sections for a given element, with various 
curves for the respective ions.  In most cases the various subshells of 
a given ion are also plotted as separate non-overlapping curves.  Resonance 
structure near threshold of outer shells is apparent, particularly in ions 
with Z$\geq$10.  The photoionization cross sections from many excited levels 
also show resonance structure.  This is illustrated in Figure 2, 
which shows a few of the excited level cross sections for O VII.  Notable are the 
resonance features near 650 eV, corresponding to the 1-2 transitions in the 
O VIII ion.  Although cross sections with comparable resolution are available 
for many ions from the opacity project, we adopt Gaussian average fits 
to these for the great majority of excited levels.  For O VII we include all 
available cross sections at high resolution for ground and excited levels 
with principle quantum number $n \leq$4 in order to illustrate the 
potential importance of the resonance structures in observed spectra.

Ground state collisional ionization rate coefficients are shown in 
Figure 3.  Each panel contains the rates for a given element as a function of 
temperature.

Figure 4 shows the radiative recombination rates we adopt.
We emphasize that these are calculated by performing 
a Milne integral (equation 3) over the photoionization cross section
for each of the bound levels of the recombined ion, and then summing over
those rates.  This is in contrast to the more typical nebular 
treatment in which such a sum is fit to an analytic formula as was 
done by, e.g. \cite{Aldrovandi1973}, and has the advantage that it 
causes all rates to go to detailed balance ratios in the proper limit.
Each panel in Figure 4 contains the rates for a given 
element.  Also shown, as the dashed curves, are the rates 
adopted in XSTAR v.1, i.e. those of \cite{Gould1970} (hydrogenic ions)
\cite{Arnaud1992} (for iron), and \cite{Aldrovandi1973} (all others).
Differences are prominent for elements such as C, O, and Fe, and 
primarily reflect differences between the previous dielectronic recombination 
rates and those adopted here (e.g. \cite{Nahar2000} and references therein).

\subsection{Ionization Balance}

In general, the state of the gas depends on both the temperature, via the 
recombination rates and collisional ionization rates, and on the radiation 
field, via the photoionization rates.  This combined dependence makes a
display of the ionization balance cumbersome in the absence of some other 
simplifying assumption.  Figure 5 shows the ionization balance in the coronal case,
i.e. when the radiation field is negligible, as a function of temperature.  This can 
be compared with other similar calculations such as, e.g. \cite{Arnaud1985}.

\subsection{Heating and cooling rates}

A by-product of the ionization and excitation balance is the emissivity and 
opacity of the gas, which correspond to the net heating and cooling rates.
Figure 6 shows the heating and cooling rates as a function 
of temperature and ionization parameter for the various elements.  
Heating rates are shown as solid curves, 
cooling rates as dashed curves.  Rates assume solar abundances
(\cite{Grevesse1996}), and are given in units of 
erg s$^{-1}$ cm$^{+3}$ per H nucleus.  Different curves 
correspond to ionization parameters log($\xi$)=0,1,2,3,4
for an $\varepsilon^{-1}$ power law ionizing spectrum
Fewer curves appear in some panels owing to pile up at low 
ionization parameters for elements such as carbon, while for 
H and He, the log$\xi \geq 2$ curves fall below the range plotted. 
These are calculated in the limit of low gas density, n=1 cm$^{-3}$.

A coronal plasma cools more efficiently, in general, than a photoionized plamsa
since the ionization state is lower at a given temperature.  Figure 7 shows the cooling rate 
as a function of temperature for such a plasma.  Comparison of these 
rates with the results of Figure 6 shows similarity with the cooling 
rate at the lowest ionization parameter plotted there (log$\xi$=0), although
the coronal rates are generally larger at low temperatures.  This is 
a reflection of the fact that at log$\xi$=0 there is significant photoionization 
of the neutral and near-neutral species.


\subsection{Thermal Balance Calculation}

When the condition of thermal equlibrium is imposed,  then the temperature
is determined as a function of ionization parameter for a photoionized plasma.
Figure 8 shows the ionization and temperature of an optically 
thin low density photoionized gas with a $\varepsilon^{-1}$ ionizing continuum,  
as a function of ionization parameter.  This can be compared with 
the results of KM82, model 7 (although that model was not optically thin 
for log($\xi$)$\leq$2).  The current model is significantly more 
highly ionized; the ionization parameter where the 
abundances of O VII and VIII reach their maximum is lower by 0.5 dex in the 
current models.   The temperature calculated here is lower; this may be due to 
a different choice of low and high energy spectral cutoffs which affect the 
Compton equilibrium temperature.  

\section{ Sample Results: High Density }

At high densities, various physical processes become important 
which can affect the ionization and thermal balance.  These include:

Lowering of the continuum, in which collisional ionization from 
highly excited levels (i.e. superlevels) results in a net reduction 
in the effective reccombination rates.  This effect is only included for 
H and He-like ions in our calculations. This process is most important 
at low temperatures.  A competing effect is collisional 
deexcitation from superlevels, but this turns out to be less important than 
continuum lowering.

3-body recombination results in a net increase in  total recombination rate.
In our models we include collisional ionization and 3-body recombination 
from essentially all levels, although this process is generally more important 
for levels closest to the continuum.

At high densities the spectroscopic level populations in the recombining ion can approach
their LTE values, leading to enhanced collisional ionization from these levels and
a decrease in the total recombination.  This turns out to be 
unimportant for most ions at the densities and temperatures we consider.

In a photoionized plasma the incident photon flux must be very large if the density 
is high and the ionization parameter is within the conventional range.  Such 
high photon fluxes can lead to large enhancements in the recombination rate
via stimulated recombination.  

\subsection{Density dependent recombination rates}

The effects of density on recombination rates 
are illustrated in figure 9, which shows recombination rate as a function 
of density for various ions of H, He and O.  The curves correspond to temperatures
logarithmically spaced between 10$^4$K and 10$^7$ K, and the dashed curve shows the 
XSTAR v.1 value.  Stimulated recombination can cause large enhancements in the rates at high 
density, but its effects are dependent on the shape of the assumed ionizing spectrum; 
therefore it has been excluded from the results until the end of
this section for illustrative purposes.  In the 
H and He-like ions the lowering of the continuum is apparent at moderate densities, and 
the effect of 3-body recombination is apparent in H I and He I at the highest densities. 
This is an illustration of the fact that, at a given density, 3 body recombination is 
greater for ions with lower free charge, and is greater at lower temperature (\cite{bautista2000}).
The calculations shown in this figure were done at a fixed ionization parameter of 
log($\xi$)=2 with an $\varepsilon^{-1}$ ionizing continuum.  As a result the lower ionization 
stages, of oxygen, O I, O II, and O III have low abundance and are not included in the full 
multilevel matrix calculation and their recombination rates are treated using the total 
rates shown in figure 3.  Other ions show the effects of continuum lowering, 
which causes a decrease in rate by a factor of up to $\sim$ a few 
beginning at densities greater than 10$^6$--10$^8$. 

\subsection{Level populations vs. density}

In addition to enhancing the total recombination rate, 
high densities enhance the importance of collisional processes relative to 
radiative processes in bound-bound transitions.  Level populations 
approach their LTE values, which may greatly exceed the recombination 
values for levels with dipole allowed decays.
This is illustrated in figure 10, which shows the ratio of level populations to 
LTE populations (departure coefficients) as functions 
of density for H I at log($\xi$)=-5  and T=10$^4$K (panel a) and for O VIII 
at log($\xi$)=-5  and T=10$^6$K (panel b).  Departure coefficients of all bound spectroscopic levels 
decrease proportional to density, approaching assymptotic values at densities
greater than 10$^{17}$ cm$^{-3}$.  The superlevels exhibit slower dependence on 
density, reflecting the fact that they are likely to be in LTE with the continuum 
at lower density than the spectroscopic levels. 

\subsection{Heating-cooling vs. density}

Heating and cooling rates depend on density via the ionization balance and 
via the rates for the heating and cooling per ion.  
In the previous subsection we have shown that at the highest densities we 
consider the recombination rates can be enhanced by 3 body recombination, or 
reduced, by continuum lowering and collisional ionization.  The former
process is dominant at high densities for H I and He I, while the latter 
dominates for moderate densities for other ions.  Since the thermal 
balance in a photoionized gas is dominated by H and He when the ionization 
parameter is low (i.e. log($\xi) \leq 1$), and by more highly charged ions 
at higher $\xi$, we expect the heating and cooling to be affected differently 
at high densities in the two different regimes.   Although the 
dependence of cooling rate on ionization balance at low densities 
is not generally monotonic (c.f. figure 6), for many ions the heating rate 
is greater at lower ionization parameter.   The per ion heating rate 
depends on the photon flux rather than the gas density, while the 
per ion cooling rate is suppressed by collisional deexcitation.
Figure 11 shows the dependence of heating and cooling rates on density and 
temperature, in a form analogous to that of figure 10.  Curves show 
cooling (dashed) and heating (solid) rates at 5 temperatures spaced 
logarithmically between between 10$^4$K and 10$^7$ K, for log($\xi$)=2
and a $\varepsilon^{-1}$ power law ionizing spectrum.  For highly ionized species
heating rates are decrease slightly with density, while cooling rates increase.
H and He I behave in the opposite way, owing to the increase in recombination
(which increases the neutral fraction and hence the photoionization heating)
and to the collisional suppression of radiative decays (which 
decreases the net radiative cooling).


\subsection{Thermal Equilibrium}

Figure 12 shows the results of a thermal balance calculation of an optically 
thin photoionized gas as a function of density and ionization 
parameter.  The curves correspond to ionization parameters log($\xi$)=4,3,2,1
for the same power law ionizing spectrum used previously.  
This demonstrates that the net effect of higher densities is 
an increase in temperature at the highest densitied and lowest temperatures,
and a decrease in temperature at lower density and higher temperature, 
for the reasons listed in the 
previous section.  We emphasize that the quantitative value of the 
temperature, particularly for high ionization parameters and/or 
temperatures greater than 10$^7$ K or so, depends on the detailed
shape of the ionizing spectrum over all energies, owing to the 
possible importance of Compton heating and cooling (the spectral 
dependence of the effects of stimulated recombination, again, have been 
excluded from these results).

\subsection{Ionization distribution, high n}

Figure 13 shows the ionization and thermal balance of an optically 
thin photoionized gas analogous to that shown in Figure 8, but 
at a density of 10$^{17}$ cm$^{-3}$.  Comparison shows that the high 
density results in generall higher ionization ste for most elements at high
ionization parameter, owing to the reduction in the net recombination 
rate.  At the same time, the temperature is slightly lower,  as described 
in the previous subsection.  The opposite is true at the low ionization parameter
extreme of figure 13 -- the temperature is slightly greater than 
in figure 9 due to the enhanced recombination rates of H and He I 
at high densities.


\subsection{Warm absorber, high density}

High densities also affect the absorption and emission 
spectra of photoionized plasmas.  Figure 14 shows a comparison 
of a warm absorber spectrum at densities of 10$^4$ cm$^{-3}$ (panel a)
and 10$^{17}$ cm$^{-3}$ (panel b) due to oxygen in the 
0.5-1 keV energy range.  The ionization parameter is log($\xi$)=2 and the 
temperature is 10$^5$K.  In the high density  case there is a prominent absorption 
structure near 0.65 keV, associated with resonances in the photoionization
cross sections from the 1s2s configurations in OVII.  These resonances are 
apparent in the cross sections shown in figure 2, and they appear in 
opacity due to the build-up of excited level populations at high densitiies.
Such features are potentially observable in the spectra of astrophysical 
X-ray sources such as the partially ionized absorbers associated 
with Seyfert galaxies (e.g. \cite{George1999}) if these objects contain 
gas at densities comparable to those considered here.

\subsection{Recombination emission, high density}

Figure 15 shows a comparison 
of the emission spectrum at densities of 10$^4$ cm$^{-3}$ (panel a)
and 10$^{17}$ cm$^{-3}$ (panel b) due to oxygen in the 
0.5-1 keV energy range.  The ionization parameter is log($\xi$)=2 and the 
temperature is 10$^5$K.  In the high density  case the
ratio of continuum to line emission is reduced, and 
the ratios of the He-like lines is changed from the familiar low-density 
case in which the forbidden/intercombination  line ratio is large 
at low density to values $\sim$1 at high density.

\subsection{The effects of stimulated recombination}

So far in this section we have artificially excluded the effects of stimulated 
recombination (by manually setting the rates to zero when calculating total 
recombination).  We illustrate the effect of relaxing this condition in figure 16, 
which is the equivalent of figure 9 (recombination rates vs. density) but 
with stimulated recombination included.  Again the ionizing spectrum is 
a $\varepsilon^{-1}$ power law, which has strong flux at the lowest photon 
energies.  Comparison of figures 16 and 9 shows that the rates are greatly enhanced 
at high densities, and this enhancement is greatest for ions with lowest 
ionization potentials.  This is due to the influence of the low energy photons 
on the stimulated recombination rate, and a different spectral shape (e.g. a blackbody) 
would produce a different distribution of recombination with charge state at high 
densities.



\newpage

\setcounter{figure}{0}

\begin{figure}
\epsfxsize=5.6in  % narrow the plot
\epsfysize=7.0in  % shorten the plot
\epsffile{fig1a.eps}
\caption{figure 1a}
\label{fig:1a}
\end{figure}


\newpage

\setcounter{figure}{0}

\begin{figure}
\epsfxsize=5.6in  % narrow the plot
\epsfysize=7.0in  % shorten the plot
\epsffile{fig1b.eps}
\caption{figure 1b}
\label{fig:1b}
\end{figure}


\newpage

\setcounter{figure}{1}

\begin{figure}
\epsfxsize=5.6in  % narrow the plot
\epsfysize=7.0in  % shorten the plot
\epsffile{fig2.eps}
\caption{figure 2}
\label{fig:2}
\end{figure}


\newpage

\setcounter{figure}{2}

\begin{figure}
\epsfxsize=5.6in  % narrow the plot
\epsfysize=7.0in  % shorten the plot
\epsffile{fig3a.eps}
\caption{figure 3a}
\label{fig:3a}
\end{figure}


\newpage

\setcounter{figure}{2}

\begin{figure}
\epsfxsize=5.6in  % narrow the plot
\epsfysize=7.0in  % shorten the plot
\epsffile{fig3b.eps}
\caption{figure 3b}
\label{fig:3b}
\end{figure}


\newpage

\setcounter{figure}{3}

\begin{figure}
\epsfxsize=5.6in  % narrow the plot
\epsfysize=7.0in  % shorten the plot
\epsffile{fig4a.eps}
\caption{figure 4a}
\label{fig:4a}
\end{figure}


\newpage

\setcounter{figure}{3}

\begin{figure}
\epsfxsize=5.6in  % narrow the plot
\epsfysize=7.0in  % shorten the plot
\epsffile{fig4b.eps}
\caption{figure 4b}
\label{fig:4b}
\end{figure}


\newpage

\setcounter{figure}{4}

\begin{figure}
\epsfxsize=5.6in  % narrow the plot
\epsfysize=7.0in  % shorten the plot
\epsffile{fig5a.eps}
\caption{figure 5a}
\label{fig:5a}
\end{figure}


\newpage

\setcounter{figure}{4}

\begin{figure}
\epsfxsize=5.6in  % narrow the plot
\epsfysize=7.0in  % shorten the plot
\epsffile{fig5b.eps}
\caption{figure 5b}
\label{fig:5b}
\end{figure}


\newpage

\setcounter{figure}{5}

\begin{figure}
\epsfxsize=5.6in  % narrow the plot
\epsfysize=7.0in  % shorten the plot
\epsffile{fig6a.eps}
\caption{figure 6a}
\label{fig:6a}
\end{figure}


\newpage

\setcounter{figure}{5}

\begin{figure}
\epsfxsize=5.6in  % narrow the plot
\epsfysize=7.0in  % shorten the plot
\epsffile{fig6b.eps}
\caption{figure 6b}
\label{fig:6b}
\end{figure}


\newpage

\setcounter{figure}{6}

\begin{figure}
\epsfxsize=5.6in  % narrow the plot
\epsfysize=7.0in  % shorten the plot
\epsffile{fig7a.eps}
\caption{figure 7a}
\label{fig:7a}
\end{figure}


\newpage

\setcounter{figure}{6}

\begin{figure}
\epsfxsize=5.6in  % narrow the plot
\epsfysize=7.0in  % shorten the plot
\epsffile{fig7b.eps}
\caption{figure 7b}
\label{fig:7b}
\end{figure}


\newpage

\setcounter{figure}{7}

\begin{figure}
\epsfxsize=5.6in  % narrow the plot
\epsfysize=7.0in  % shorten the plot
\epsffile{fig8a.eps}
\caption{figure 8a}
\label{fig:8a}
\end{figure}


\newpage

\setcounter{figure}{7}

\begin{figure}
\epsfxsize=5.6in  % narrow the plot
\epsfysize=7.0in  % shorten the plot
\epsffile{fig8b.eps}
\caption{figure 8b}
\label{fig:8b}
\end{figure}


\newpage

\setcounter{figure}{8}

\begin{figure}
\epsfxsize=5.6in  % narrow the plot
\epsfysize=7.0in  % shorten the plot
\epsffile{fig9a.eps}
\caption{figure 9a}
\label{fig:9a}
\end{figure}


\newpage

\setcounter{figure}{8}

\begin{figure}
\epsfxsize=5.6in  % narrow the plot
\epsfysize=7.0in  % shorten the plot
\epsffile{fig9b.eps}
\caption{figure 9b}
\label{fig:9b}
\end{figure}

