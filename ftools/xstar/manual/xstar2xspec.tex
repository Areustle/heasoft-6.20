\chapter{XSTAR2XSPEC}
\label{sec:xstar2xspec}

To facilitate using XSTAR with actual data, XSTAR2XSPEC was 
developed.  XSTAR2XSPEC is a perl script which calls XSTAR multiple times 
and generates table models from the results of these simulations 
which can then be utilized for model fitting in the XSPEC spectral 
fitting program.  XSTAR2XSPEC also generates a log file which is a concatenation 
of all the xout\_step.log files from all the XSTAR runs.  This is a convenient  
way of generating a grid of models for other purposes, such as studying 
the dependence of line strengths on various input parameters, or exploring 
the full dependence of the heating and cooling rates on temperature 
and ionization parameter.  This last problem is illustrated in the 
examples section of this chapter.

\section{Parameters}

The parameter handling for XSTAR2XSPEC is designed to be as flexible 
as possible, in principle, limited only by the physical resources 
(RAM, disk space \& CPU time) available on your machine.  You have the choice of 
varying {\it any} of the physical parameters used as input in an XSTAR 
model.

\subsection{Physical Parameters}

XSTAR models are based on 21 physical parameters described in detail 
in Chapter~\ref{sec:xstarinput}.  In quick summary, they are 
cfrac, temperature, pressure, density, trad, rlrad38, column, rlogxi,
habund, heabund, cabund, nabund, oabund, neabund, mgabund, siabund, sabund, 
arabund, caabund, feabund \& niabund.  Each of these parameters needs 
at least one additional parameters and as many as four additional 
parameters to specify is variation during the program run.

Each parameter has three levels for classifying its variability:

\begin{description}
	\item[{\bf Constant (variation type = 0):}]  This parameter is held 
	constant in all the XSTAR runs.

	\item[{\bf Additive (variation type = 1):}]  The Additive class of 
	parameters provides a 
	simple method for varying parameters that are reasonably 
	independent of the others.  For more info on how additive parameters 
	function, see the XSPEC manual and OGIP Memo OGIP 92-009.

	\item[{\bf Interpolated (variation type = 2):}]  Interpolated parameters 
	provide the 
	greatest accuracy in building table models.  They also require the 
	most processing time.  For each interpolated parameter, you can 
	define some number of points between a maximum and minimum range.  
	The placement of these intermediate points is determined by the 
	interpolation type -- linear (interpolation type = 0) or 
	logarithmic (interpolation type = 1).
	
\end{description}

In estimating the running time of XSTAR2XPEC, the key factor is the 
number of times XSTAR is called.  Consider a run with ${\rm N}_{\rm I}$ 
interpolated parameters, where interpolated parameter $i$ is evaluated 
at $n_{i}$ points $(1 \leq i \leq {\rm N}_{\rm I})$.  The total number of 
times XSTAR is called is then $\prod\limits_{\rm i=1}^{{\rm N}_{\rm 
I}} n_i$.  However, if ${\rm N}_{\rm A}$ additive parameters are also 
defined, then for each set of interpolated variables, there is one 
run with all the additive parameters are zero and the remaining ${\rm 
N}_{\rm A}$ runs have one of the additive parameters at it's maximum 
value and the rest all zero.  This means that the total number of 
times XSTAR must be run is given by

\begin{equation}
	({\rm N}_{\rm A}+1) \prod\limits_{\rm i=1}^{{\rm N}_{\rm I}} n_i
\end{equation}
and this can provide you with a feel for how long a complete 
XSTAR2XSPEC run will require.  As an example, if we defined 2 
interpolated parameters (one evaluated at 5 points and the other 
evaluated at 4 points) and 8 additive parameters, XSTAR would be run a 
total of
\begin{equation}
	(8+1)(5)(4) = 180 calls
\end{equation}
which means that if the XSTAR runs for the appropriate model range 
averages five minutes, it will take approximately (180)(5 minutes) = 
900 minutes = 15 hours for a complete XSTAR2XSPEC run.

\subsection{XSTAR Fixed Parameters}

These values are the same for {\it all} XSTAR runs.

\subsection{XSTAR2XSPEC Control parameters}

\begin{description}
	\item[{\bf elow:}]  This parameter determines the low energy end (in 
	eV) of the spectrum selected from the XSTAR output files.

	\item[{\bf ehigh:}]   This parameter determines the high energy end (in 
	eV) of the spectrum selected from the XSTAR output files.
\end{description}

\section{Running XSTAR2XSPEC}

To run XSTAR2XSPEC, simply type

\begin{verbatim}
	xstar2xspec [options]
\end{verbatim}
at the system prompt, where [options] consists of one or more of the 
options listed below.  Generally, you will run XSTAR2XSPEC with no 
options.  There are a couple of control options which are 
described below.  Currently the script prompts you for {\it all} 
parameters by invoking the pset utility (included in the standard 
XSTAR package).  The defaults are adequate for most situations.

Entering all the many parameters prompted by XSTAR2XSPEC can be 
tedious, and prone to error.  An alternative is to edit the parameter 
file using an FTOOL routine or a text editor.  Since XSTAR2XSPEC is 
actually a perl script and not an FTOOL, it does not have its own 
parameter file.  Rather, it calls several other FTOOLS which do have 
their own parameter files.  The FTOOL responsible for setting up the 
multiple commands to call XSTAR is called XSTINITABLE, and all 
the relevant parameters can be set by editing the parameter file
xstinitable.par.  This is what we do when using this script.

{\bf XSTAR2XSPEC Options:}

\begin{description}

	\item[{\tt -save}:]   Save the spectral FITS files, modifying the file name to
    include the value of the loopcontrol variable for better
    identification.  Note this can use GBs of disk space.
    
	\item[{\tt -verbose}:]   Generate more diagnostic messages (applies 
	to the XSTAR2XSPEC script only).

	\item[{\tt -restart}:]  Continues the XSTAR2XSPEC run using the previous run.
    Note that it does {\it NOT} check the integrity of the table files from the 
    terminated run.  This is the user's responsibility.
\end{description}

\subsection{Output}

Four files are produced by xstar2xspec:

\begin{enumerate}
	\item xstar2xspec.log is a concatenation of all the xout\_step.log
	files from all the xstar runs called by xstar2xspec.

	\item xout\_ain.fits is a fits file containing the atable with the reflected 
	emission spectrum produced by xstar.  This file is ready for use by xspec
	using the `model atable xout\_ain.fits ...' command.

	\item xout\_aout.fits is fits file containing the atable with the
	emission spectrum in the forward (transmitted) direction 
	produced by xstar.  This file is ready for use by xspec
	using the `model atable xout\_aout.fits ...' command.

	\item xout\_mtable.fits is a fits file containing the mtable with the 
	absorption spectrum in the forward (transmitted) direction
	produced by xstar.  This file is ready for use by xspec
	using the `model mtable xout\_mtable.fits pow ...' command.

\end{enumerate}

\subsection{Important Notes on Mtables}

If you are using mtables with variable abundances, then you will likely get 
totally unphysical results unless you note the following.  

The mtable models are constructed using two kinds of parameters:  interpolated 
parameters and additive parameters.  Interpolated parameters are treated 
as one might expect:  if the model is represented by a vector $M_i(x_j)$, 
corresponding to the model flux at various energies $\varepsilon_i$ and 
stored at various values of the free (intepolated) parameter $x_j$, 
then for a value of the parameter not on the tabulated grid xspec 
calculates the model value as

$$M_i(x)=\Sigma_j M_i(x_j)\omega_j$$

\noindent where $\omega_j$ are suitably chosen weights for, eg. linear 
interpolation.  Ths is in contrast to the treatment of additive parameters:
for an mtable, xspec needs the value of the model tabulated at only 
two values of the free additive parameter $y$, 0 and $y_{max}$.  Then the 
model is calculated for some arbitrary $y$ as

$$M_i(y)=(M_i(y_{max})-M_i(0)){{y}\over{y_{max}}}+M_i(0)$$

\noindent i.e. it is assumed that the model scales linearly with the value 
of the additive parameter $y$.  This formalism was developed for 
emission models, where emissivities might be expected to scale 
linearly with elemental abundance.  In the case of absorption models, 
the value of $M$ which xspec uses is a transmission coefficient, and this 
does not scale linearly with abundance.  Rather, the transmission coefficient 
is related to the optical depth by:

$$M_i(y)=e^{-\tau_i(y)}$$

\noindent and the optical depth $\tau_i(y)$ does scale (approximately) linearly
with abundance.  For this reason, xspec has incorporated the etable models, 
in which the model builder supplies optical depths rather than transmission 
coefficient, linear interpolation is used to calculate the optical depth 
for a given abundance, and then the transmission coefficient is 
calculated using the above expression.  The results:

(1) If you use an mtable calculated by xstar2xspec using varable abundances
(i.e. any additive parameters allowed to vary) in xspec using 
the `model mtable{xout\_mtable.fits}' command, you will get unphysical 
results.  It is easy to see that if $M_i(0)$ is non-zero (which it often 
is owing to 0 abundances) then you will never get deep absorption.

(2) If so, you should use the `model etable{xout\_mtable2.fits}' command
in xspec instead.  Entering parameter, etc., in xspec is the same for 
etables as for mtables.

(3) BUT, the etable itself is supposed to be filled with optical depths, not
transmission coefficients.  The tables made by xstar2xspec are transmission coefficients,
and they need to be converted.  This is a simple transformation:  replace 
the contents of every cell of the model extension of the mtable file ($x_{ij}$)
with -ln($x_{ij}$).  This can be done various ways.  One way is to use the calculator 
inside of the ftools gui routine xv, one column at a time.  This has 
been done for the various precalculated models on the website, and these 
are stored with the names xout\_mtable2.fits.


\begin{verbatim}
_________________________________________________________________________________
\end{verbatim}

\noindent{\bf Some Specific Examples, and a Discussion of the Analytic Model warmabs}

When fitting to absorption spectra it is important to 
be aware of  the inherent limitations of xstar when applied to the 
fitting of absorption.  There are two distinct reasons for this:
spectral resolution effects, and interpolation effects.
In order to illustrate these effects, we reproduce 
here a discussion between TK and P. Oneill of Imperial College in 
which these questions are raised, and some of the issues are discussed.

Question:

I've been using some of the sample XSPEC models from the current version 
of XSTAR to model a warm absorber.

I've been using the grid18 models with the abundances fixed at their 
default values. I found that using the XSPEC model 
"mtable{xout\_mtable.fits}*powerlaw" gave different results (less 
absorption) than when I used "etable{xout\_mtable\_2.fits}*powerlaw".  I 
thought that these should provide the same results, so long as the 
abundances are not changed. Am I using these tables correctly, or is 
there perhaps a problem somewhere?

Answer:

The problem is that variable abundances, they are treated as additive
parameters in xspec tables, really can't be treated 
accurately  using either type of table.  If you use an mtable, then 
the transmittivity in a bin, i, at a given ionization parameter and 
column, is:

\begin{equation}
M_i=M_i^0+\Sigma_j x_j M_i^j
\end{equation}

where $M_i^0$ is the 'zero abundance' version of the model at that 
ionization parameter and column, $x_j$ is the abundance of element j, 
and $M_i^j$ is the model calculated with that element set to unity (relative 
to cosmic).  The first problem with this approach is that tranmission 
doesn't  scale this way with abundance. If you double the abundance 
of an element, you should double the optical depth.  Doubling the 
transmittivity has both the wrong qualitative and quantitative behavior.
But an even bigger problem with this is that the whole thing, $M_i$, 
multiplies your continuum.  It is almost guaranteed to bigger than 1 if 
some of the $x_j$ are non-zero.  So this is totally the wrong formulation to 
use when the abundances vary.  


In the case of an etable the transmittivity is:

\begin{equation}
M_i=exp(-(E_i^0+\Sigma_j x_j E_i^j))
\end{equation}

where $E_i^0=-{\rm ln}(M_i^0)$ etc.

This is better, since it predicts the correct dependence on abundance,
i.e. that the optical depth scales approximately linearly with 
abundance.

But it is important to remember that what you are trying 
to simulate is a 'real' photoionized gas with some abundance set {$x_j$} 
that fits your data.  What you are using to fit to your data are $M_i^0$, 
which in the case of xstar are models calculated with only hydrogen and 
helium, and the set of $M_i^j$, which are supposed to be calculated with 
pure element j.  

The first  problem is that the cooling, and therefore the 
temperature, depends on the element abundance in the gas, and this is not 
linear.  The temperature couples back into the ionization balance, and 
that affects the opacity, transmittivity, etc. 

The second problem is that, while it is straightforward to calculate 
a model with no elements heavier than H and He, 
it is not straightforward to calculate a 
model with pure iron, or calcium, say.  That's because H and He havs nice 
simple cooling properties at low temperature, and their opacity is simple, 
etc.  Pure iron models at best are likely to be very different than 
H+He+Fe models, or from a cosmic mix.  So what I do is  make the $M_i^j$ 
using H+He+element j in cosmic ratios.  But the problem with this is that 
then the opacity due to H and He are present 
{\em in every $M_i^j$, and $E_i^j$}, 
particularly at low ionization parameters.  So if you sum over cosmic 
abundances using an etable you are including the H and He opacity with 
every one of the $E_i^j$.  At high ionization parameter (log($\xi)>$1?) 
this 
should not be a problem because H and He are ionized sufficiently that they do 
not contribute to the opacity.  But at low ionization parameter, the 
model spectra calculated using etables will have significantly more 
low energy opacity due to the multiple counting of H and He than they 
would in a real xstar model with the corresponding parameters.  

The strategy to use at low ionization parameters therefore
is to use the etable to find a reasonably good fit to the 
spectrum with xspec, and then run xstar2xspec and make a very small table, 
i.e. just 1 column and 2 ionization parameters (xspec needs tables to 
have at least 2 entries), with  constant abundances, set to the values 
which come from the fit.  Constant abundance grids (such as grid19) when used 
as mtables do not suffer from any of these problems.

I attach 4 figures to illustrate these problems.  All use the mtable or 
etable from grid18.  All are plotted between 0.6 and 0.7 keV, power law 
continuum, index=1, norm=1, log($\xi$)=1.  The continuum level at .6 keV is 1.66.



\begin{figure}
\epsfxsize=5.6in  % narrow the plot
\epsfysize=7.0in  % shorten the plot
\epsffile{warmabsfig1.epsi}
\caption{figure 1: cosmic abundances, mtable}
\label{fig:1 cosmic abundances, mtable}
\end{figure}


\begin{figure}
\epsfxsize=5.6in  % narrow the plot
\epsfysize=7.0in  % shorten the plot
\epsffile{warmabsfig2.epsi}
\caption{figure 2: cosmic abundances, etable}
\label{fig:2 cosmic abundances, etable}
\end{figure}


Comparing these shows the big problem with mtables with variable abundances:  they give a 
transmittivity which is $>$1.  In this case, the transmittivity is $>$10.

\begin{figure}
\epsfxsize=5.6in  % narrow the plot
\epsfysize=7.0in  % shorten the plot
\epsffile{warmabsfig3.epsi}
\caption{figure 3: oxygen=1, other metals=0, etable}
\label{fig:3 oxygen=1, other metals=0, etable}
\end{figure}

\begin{figure}
\epsfxsize=5.6in  % narrow the plot
\epsfysize=7.0in  % shorten the plot
\epsffile{warmabsfig4.epsi}
\caption{figure 4: oxygen=1, other metals=0, mtable}
\label{fig:4 oxygen=1, other metals=0, mtable}
\end{figure}


Here again, the problem is that even the pure oxygen mtable has the 
transmittivity of the 'zero abundance' model, $M_i^0$, which really has pure 
H and He and therefore has unit transmittivity.  The line never can go 
black.  The only model which is nearly correct is 3, but here again you 
have the problem of multiple counting of the H and He opacity.  But at 
this high ionization parameter that is negligible.  Anyway, I hope you 
understand the problem. 

It is straightforward to use xstar2xspec to calculate grids without 
variable abundances.  THEN you can use mtables, and there should be none 
of these problems.


Question:

I have made some comparisons between vturb=0 models and 
those with vturb=100.  I find that vturb=0  gave very different results 
(much less absorption) to the vturb=100 grid18 etable.  I 
hadn't expected to see the large difference with a change from vturb=100 
to vturb=0. Is this expected behaviour? 

Answer:

vturb can make a significant difference in the results of xstar modeling 
of absorption spectra, and the reason is at least 
partially due to numerics.  In calculating the energy dependent opacity  Xstar simply 
evaluates the profile function for the line at each grid energy.  So 
if the line width is less than, or comparable to, the xstar grid spacing, 
then maximum depth of a given absorption line in the mtable  may be very 
different from the true depth at line center.  Obviously this problem is 
more severe for small vturb, i.e. narrow lines.  The total line equivalent 
width does not depend on vturb, but the numerics do not reflect that 
unless the line is broader than the grid resolution.  And the limiting 
resolution is the internal xstar grid spacing, which corresponds to 420 
km/s, since this is what is used in constructing the table.  This is 
another flaw in the use of tables for fitting to spectra.   

\begin{verbatim}
_________________________________________________________________________________
\end{verbatim}

There is now subroutine which calculates xstar warm absorber (called 'warmabs') and 
warm emitter (called 'photemis') spectra and which can be called as 
an xspec `analytic' model.  This calculates absorption spectra
`on the fly', and so  will evaluate the profile 
function on whatever grid you use in xspec.  So you can choose to resolve 
all the lines if you want, using the 'dummyrsp' command with a very large number 
of bins.  It is available on the xstar web site, look for the link under `XSTAR news'.
It also does not suffer from the problems discussed previously about the 
approximations inherent in variable abundance.  It calculates the 
opacity directly from a stored library of level populations calculated for a
generic power law grid.  

One drawback of this routine is that it brute-force 
calculates Voigt profiles and opacity for all the lines and bound-free
continua which are above a certain threshold in strength, and so it is time-consuming.
Currently calls to this routine can take up to 30 seconds on a slow or busy machine. 
It will be much faster on a faster machine, or if there are few elements with non-zero 
abundances, or if only a narrow spectral range is of interest, or if the 
ionization parameter is frozen.

Another drawback to this routine is that the current version assumes 
constant ionization throughout a slab of given ionization parameter and column.
Real slabs will have lower mean ionization in the deeper, shielded regions, and 
this lower ionization material will absorb more efficiently.
This means that the current warmabs assumption results in an 
underestimate of the opacity, particularly at low energies,
for slabs with large column.  This will be remedied in future versions of this routine.


\begin{verbatim}
_________________________________________________________________________________
\end{verbatim}

Yet another issue which deserves mention is that of interpolation and grid 
spacing in the interpolated parameters as implemented by xspec.  Many of the 
available xstar grids were calculated with models spaced 1 decade apart in column.
This has the potential to lead to inaccuracies in the model spectrum calculated 
by xspec, since it will interpolate between grid points.  Sparsely gridded 
models may miss important details.  Grids 19b and 19c provide an example.


\begin{verbatim}
_________________________________________________________________________________
\end{verbatim}

Question:

I have compared the grids with the warmabs analytical model. These are in 
firgure 5. 19c is the solid line, warmabs is the dashed line, and 19b 
is the dotted line. The model here is the same model that I used to fit 
the data using 19c (ie, it has 1.4e20 of wabs also). I hope you can get 
some info from this. The parameters used for these models were: log$\xi$=1.27, 
N=5.89e22.

\begin{figure}
\epsfxsize=5.6in  % narrow the plot
\epsfysize=7.0in  % shorten the plot
\epsffile{warmabsfig5.epsi}
\caption{figure 5: Comparison of grid19b, 19c, and warmabs}
\label{fig:Comparison of grid19b, 19c, and warmabs}
\end{figure}

Answer:

OK, so what I see from your plot comparing the models is that:

1) grid19b and grid19c give qualitatively different results near the best 
fit parameter values in the strength of the absorption near the K edge of 
oxygen.  This seems to me to be clear evidence for the errors introduced 
by interpolation in a sparsely gridded table in column, as I suggested. 

2) grid19c and warmabs agree well at energies above .8 keV, except for the 
lines.  This seems to me to be a comforting check on the validity of 
warmabs.  

3) there is disagreement in the strength of the absorption below 0.8 keV.  
We decided that the assumption of a constant ionization slab made in 
warmabs would tend to underestimate the absorption, since the ionization 
balance would be lower in the shielded parts of a real slab, hence more 
absorption.  This is the sense of the disagreement, and so seems to make 
sense.  Of course, there is still some interpolation involved in the use 
of grid19c, since your best fit values for N and $/xi$ are not precisely on 
the grid values.  I have compared warmabs and grid19c for the nearest grid 
values, and the effect of interpolation does not appear to change the 
sense of the disagreement.  So this is a strong argument for fixing up 
warmabs so that it does not make the uniform slab assumption.

4) There is also disagreement between grid19c and warmabs about the 
strength of the lines.  I think this is due to the effect of energy 
binning:  Your energy grid must be rather coarse, i.e.R=E/Delta(E)~100.
If you make the same plot with a grid resolution of R=10000, things look 
better.  I attach these plots: figure 6=warmabs, figure 7=grid19c.  But there is 
still disagreement, and this I think is because the grid19c results have 
the intrinsic xstar internal resolution, which is much less than 10000, 
and so information is lost about the line profiles, while warmabs 
calculates the lines specifically for the resolution requested, and so 
should be more accurate.

\setcounter{figure}{6}

\begin{figure}
\epsfxsize=5.6in  % narrow the plot
\epsfysize=7.0in  % shorten the plot
\epsffile{warmabsfig6.epsi}
\caption{figure 6: warmabs model, same parameters as figure 5, R=10000}
\label{fig:6 }
\end{figure}

\begin{figure}
\epsfxsize=5.6in  % narrow the plot
\epsfysize=7.0in  % shorten the plot
\epsffile{warmabsfig7.epsi}
\caption{figure 7: grid19c, same parameters as figure 5, R=10000}
\label{fig: 7 }
\end{figure}


\subsection{Notes on Normalization}


The problem with creating a flexible tool for modelling emission and 
absorption is that there have several free parameters affecting 
real spectra, including: source luminosity, distance, reprocessor column density, 
ionization parameter, and geometry.  By geometry we mean 
covering fraction around the source, which affects emission, and covering
fraction across our line of sight to the source, which affects absorption.
With the xstar2xspec tables you should be able to model a wide range
of choices for this, but there is not a unique one-to-one mapping between the 
values of these physical parameters and the values used in running xstar and 
constructing the tables.

The free parameters which can be varied when running xstar2xspec include the abundances, 
column density, gas density, and ionization parameter.  These all have a straightforward intepretation 
as physical parameters.  The emitter normalization and geometry are not 
uniquely determined, owing to the ambiguity between source luminosity and distance. 

It is helpful here to be very specific, at the risk of being repetitive.
The procedure followed by xstar2xspec in making tables is:  (i) Generate 
a sequence of command line calls to xstar (this is done by the tool xstinitable);
(ii) step through the calls to xstar; (iii) for each one calculate the appropriate 
xstar model; (iv) and then  take the spectrum output of the model (the file xout\_spect1.fits) and 
convert it to the right units and append it to the fits table (xout\_aout.fits, xout\_ain.fits,
or xout\_mtable.fits).  The last step (iv) is done by the tool xstar2table.  The conversion 
is as follows:  xspec wants a binned spectrum in units model counts/bin for atables.
If this is denoted $F_n^{mod}$, and the luminosity used in calculating the xstar grid is $L_{tot}^{xstar}$
then

$$F_n^{mod}=\frac{L_\varepsilon^{xstar}}{L_{tot}^{xstar}} \frac{10^{38}}{4 \pi (1 {\rm kpc})^2} \left(\frac{\Delta\varepsilon}{\varepsilon}\right)$$

which can be rewritten:

$$F_n^{mod}=\frac{L_\varepsilon^{xstar}}{L_{tot}^{xstar}} \left(\frac{\Delta\varepsilon}{\varepsilon}\right) 8.356 \times 10^{-7}$$

and $\left(\frac{\Delta\varepsilon}{\varepsilon}\right)$ is the fractional energy bin size.  

The meaning of this quantity and the normalization can be better understood if we consider how 
xspec works in more detail.  Xspec calculates the model count rates per bin by multiplying the 
$F_n^{mod}$ vector with the response matrix $A_{nm}$ and multiplying by a normalization factor $\kappa$:

$$C_m^{mod}=\kappa \Sigma_n{F_n^{mod} A_{nm}}$$

\noindent The observed count rate $C_m^{obs}$ is calculated from the physical flux recieved by the 
satellite $F_\varepsilon^{obs}$:

$$C_m^{obs}=\Sigma_n{F_\varepsilon^{obs} A_{nm}}\left(\frac{\Delta\varepsilon}{\varepsilon}\right)$$

\noindent Then it's easy to see that $C_m^{obs}=C_m^{mod}$ if the shape of the model fits the observations and

$$\kappa = \frac{C_m^{obs}}{C_m^{mod}}$$

or

$$\kappa=\frac{F_\varepsilon^{obs} \left(\frac{\Delta\varepsilon}{\varepsilon}\right)}{F_n^{mod}}$$

Now, if the astronomical X-ray source actually resembles the physical scenario assumed by xstar, i.e. if it 
consists of a shell of photoionized material surrounding
a point source of continuum with covering fraction $f$  then $F_\varepsilon^{obs}=f L_\varepsilon^{source}/(4\pi D^2)$, where 
$L_\varepsilon^{source}$ is the actual specific luminosity emitted by the shell, and $D$ is the distance 
to the source.  Then

$$\kappa=f \frac{L_\varepsilon^{source}/10^{38}}{L_\varepsilon^{xstar}/L_{tot}^{xstar}} \frac{1}{D_{kpc}^2}$$

And, if we have gotten the model exactly right and $L_\varepsilon^{source}=L_\varepsilon^{xstar}$ 
(which implies that $L_{tot}^{source}=L_{tot}^{xstar}$) then

$$\kappa=f  \frac{L_{tot}^{xstar}/10^{38}}{D_{kpc}^2}$$

If, on the other hand, the shape of the emitted spectrum is right but the luminosity of the astrophyscial source is 
different from the luminosity used in calculating the xstar model then 
$L_\varepsilon^{source}=L_\varepsilon^{xstar} L_{tot}^{source}/L_{tot}^{xstar}$
and 

$$\kappa=f  \frac{L_{tot}^{source}/10^{38}}{ D_{kpc}^2}$$

which can be inverted to find the things on the right hand side if you have a fitted value for $\kappa$.



So, for example, if your atable model fits to the data with a normalization=1, let's say,
and you used luminosity=1 in creating the table, then this would
imply that your data was consistent with a full shell illuminated 
by a luminosity of 10$^{38}$ erg/s at a distance of 1 kpc, 
or it also could be a shell illuminated by a luminosity of 10$^{44}$ erg/s 
at a distance of 1000 kpc.  

For another example, let's say  you know the distance to the source 
is 1 kpc and the luminosity is 10$^{38}$, but the best fit has an emitter 
normalization of 0.1.  This would suggest (to me) that rather than 
a full sphere, the emitter only subtends 10$\%$ of the solid angle around the 
source.  

Obviously, the column density of the emitter is important also.  If you 
don't know the column density, then a shell of column density 10$^{19}$ cm$^{-2}$ 
illuminated by a 10$^{38}$ erg s$^{-1}$ source will probably have very
similar emitted X-ray spectrum to a shell of column density 10$^{20}$ illuminated by a 
10$^{37}$ erg s$^{-1}$ source.  If there are absorption features in the 
spectrum, then they may constrain the column density.


\subsection{Speeding Things Up}

If your disk or CPU resources are limited, you might want methods to 
reduce the execution time of and XSTAR run.  Here are some methods:

\begin{enumerate}
	\item  In most cases, you are interested in the physical conditions 
	in a plasma of fixed composition (usually solar).  In this case, you 
	can define the variation type of the composition parameters as 
	zero.  This will keep the abundances constant and can reduce the 
	running time by about a factor of twelve.

	\item  Keep the number of interpolated parameters at a {\it 
	minimum}.  Two is usually sufficient.  Two interpolated parameters 
	sampled at five points each requires 25 runs of XSTAR.  Adding two 
	more parameters at the same sampling requires 625 XSTAR runs.  With 
	our sample five minute (optimistically) XSTAR run, that comes out 
	to over 52 {\it hours}!

\end{enumerate}

\section{Examples}

In what follows we give a couple of examples of the use of XSTAR2XSPEC.  
Here we provide the entire parameter file XSTINITABLE.PAR.  If the 
desired application resembles one of these, then the user can edit these 
files and copy them into the pfiles directory.

\subsection{Example 1: A grid of coronal models}

In this example density is held constant, column density is low, 
thermal equilibrium is not satisfied, and temperature and 
element abundances are varied in the manner familiar from 
models such as APEC or MEKAL.

\begin{verbatim}

cfrac,r,a,1.,0.,1.,"covering fraction soft maximum"
cfractyp,i,h,0,0,2,"covering fraction variation type"
cfracint,i,a,1,0,1,"covering fraction interpolation type"
cfracsof,r,a,0.,0.,1.,"covering fraction soft minimum"
cfracnst,i,a,1,1,20,"covering fraction number of steps"
temperature,r,h,1000.,0.,1.E4,"temperature soft maximum (/10**4K)"
temperaturetyp,i,h,2,0,2,"temperature variation type"
temperatureint,i,a,1,0,1,"temperature interpolation type"
temperaturesof,r,a,1.,0.,1.,"temperature soft minimum"
temperaturenst,i,a,7,1,20,"temperature number of steps"
pressure,r,h,0.03,0.,1.,"pressure soft maximum (dyne/cm**2)"
pressuretyp,i,h,0,0,2,"pressure variation type"
pressureint,i,a,1,0,1,"pressure interpolation type"
pressuresof,r,a,0.,0.,1.,"pressure soft minimum"
pressurenst,i,a,1,1,20,"pressure number of steps"
density,r,a,1.E+8,0.,1.E18,"density soft maximum (cm**-3)"
densitytyp,i,h,0,0,2,"density variation type"
densityint,i,a,0,0,1,"density interpolation type"
densitysof,r,a,1.e+10,0.,1.e+18,"density soft minimum"
densitynst,i,a,2,1,20,"density number of steps"
trad,r,a,-1.,,,"radiation temperature or alpha soft maximum?"
tradtyp,i,h,0,0,2,"radiation temperature variation type"
tradint,i,a,1,0,1,"radiation temperature interpolation type"
tradsof,r,a,0.,0.,1.,"radiation temperature soft minimum"
tradnst,i,a,1,1,20,"radiation temperature number of steps"
rlrad38,r,a,1.,0.,1.E10,"luminosity soft maximum (/10**38 erg/s)"
rlrad38typ,i,h,0,0,2,"luminosity variation type"
rlrad38int,i,a,1,0,1,"luminosity interpolation type"
rlrad38sof,r,a,0.,0.,1.,"luminosity soft minimum"
rlrad38nst,i,a,1,1,20,"luminosity number of steps"
column,r,a,1.E17,0.,1.E25,"column density soft maximum (cm**-2)"
columntyp,i,h,0,0,2,"column density variation type"
columnint,i,a,1,0,1,"column density interpolation type"
columnsof,r,a,1.E17,1.,1.E25,"column density soft minimum"
columnnst,i,a,1,1,20,"column density number of steps"
rlogxi,r,a,-6.0,-10.,+10.,"log(ionization parameter) soft maximum (erg cm/s)"
rlogxityp,i,h,0,0,2,"log(ionization parameter) variation type"
rlogxiint,i,a,0,0,1,"log(ionization parameter) interpolation type"
rlogxisof,r,a,-6.,-10.0,+10.0,"log(ionization parameter) soft minimum"
rlogxinst,i,a,0,1,20,"log(ionization parameter) number of steps"
habund,r,h,1.,0.,100.,"hydrogen abundance soft maximum"
habundtyp,i,h,0,0,2,"hydrogen abundance variation type"
habundint,i,a,1,0,1,"hydrogen abundance interpolation type"
habundsof,r,a,0.,0.,1.,"hydrogen abundance soft minimum"
habundnst,i,a,1,1,20,"hydrogen abundance number of steps"
heabund,r,h,1.,0.,100.,"helium abundance soft maximum"
heabundtyp,i,h,0,0,2,"helium abundance variation type"
heabundint,i,a,1,0,1,"helium abundance interpolation type"
heabundsof,r,a,0.,0.,1.,"helium abundance soft minimum"
heabundnst,i,a,1,1,20,"helium abundance number of steps"
liabund,r,h,0.,0.,100.,"lithium abundance soft maximum"
liabundtyp,i,h,0,0,2,"lithium abundance variation type"
liabundint,i,h,1,0,1,"lithium abundance interpolation type"
liabundsof,r,h,0.,0.,1.,"lithium abundance soft minimum"
liabundnst,i,h,1,1,20,"lithium abundance number of steps"
beabund,r,h,0.,0.,100.,"beryllium abundance soft maximum"
beabundtyp,i,h,0,0,2,"beryllium abundance variation type"
beabundint,i,h,1,0,1,"beryllium abundance interpolation type"
beabundsof,r,h,0.,0.,1.,"beryllium abundance soft minimum"
beabundnst,i,h,1,1,20,"beryllium abundance number of steps"
babund,r,h,0.,0.,100.,"boron abundance soft maximum"
babundtyp,i,h,0,0,2,"boron abundance variation type"
babundint,i,h,1,0,1,"boron abundance interpolation type"
babundsof,r,h,0.,0.,1.,"boron abundance soft minimum"
babundnst,i,h,1,1,20,"boron abundance number of steps"
cabund,r,h,1.,0.,100.,"carbon abundance soft maximum"
cabundtyp,i,h,1,0,2,"carbon abundance variation type"
cabundint,i,a,1,0,1,"carbon abundance interpolation type"
cabundsof,r,a,0.,0.,1.,"carbon abundance soft minimum"
cabundnst,i,a,1,1,20,"carbon abundance number of steps"
nabund,r,h,1.,0.,100.,"nitrogen abundance soft maximum"
nabundtyp,i,h,1,0,2,"nitrogen abundance variation type"
nabundint,i,a,1,0,1,"nitrogen abundance interpolation type"
nabundsof,r,a,0.,0.,1.,"nitrogen abundance soft minimum"
nabundnst,i,a,1,1,20,"nitrogen abundance number of steps"
oabund,r,h,1.,0.,100.,"oxygen abundance soft maximum"
oabundtyp,i,h,1,0,2,"oxygen abundance variation type"
oabundint,i,a,1,0,1,"oxygen abundance interpolation type"
oabundsof,r,a,0.,0.,1.,"oxygen abundance soft minimum"
oabundnst,i,a,1,1,20,"oxygen abundance number of steps"
fabund,r,h,0.,0.,100.,"fluorine abundance soft maximum"
fabundtyp,i,h,0,0,2,"fluorine abundance variation type"
fabundint,i,h,1,0,1,"fluorine abundance interpolation type"
fabundsof,r,h,0.,0.,1.,"fluorine abundance soft minimum"
fabundnst,i,h,1,1,20,"fluorine abundance number of steps"
neabund,r,h,1.,0.,100.,"neon abundance soft maximum"
neabundtyp,i,h,0,0,2,"neon abundance variation type"
neabundint,i,a,1,0,1,"neon abundance interpolation type"
neabundsof,r,a,0.,0.,1.,"neon abundance soft minimum"
neabundnst,i,a,1,1,20,"neon abundance number of steps"
naabund,r,h,0.,0.,100.,"sodium abundance soft maximum"
naabundtyp,i,h,0,0,2,"sodium abundance variation type"
naabundint,i,h,1,0,1,"sodium abundance interpolation type"
naabundsof,r,h,0.,0.,1.,"sodium abundance soft minimum"
naabundnst,i,h,1,1,20,"sodium abundance number of steps"
mgabund,r,h,1.,0.,100.,"magnesium abundance soft maximum"
mgabundtyp,i,h,1,0,2,"magnesium abundance variation type"
mgabundint,i,a,1,0,1,"magnesium abundance interpolation type"
mgabundsof,r,a,0.,0.,1.,"magnesium abundance soft minimum"
mgabundnst,i,a,1,1,20,"magnesium abundance number of steps"
alabund,r,h,0.,0.,100.,"aluminium abundance soft maximum"
alabundtyp,i,h,0,0,2,"aluminium abundance variation type"
alabundint,i,h,1,0,1,"aluminium abundance interpolation type"
alabundsof,r,h,0.,0.,1.,"aluminium abundance soft minimum"
alabundnst,i,h,1,1,20,"aluminium abundance number of steps"
siabund,r,h,1.,0.,100.,"silicon abundance soft maximum"
siabundtyp,i,h,1,0,2,"silicon abundance variation type"
siabundint,i,a,1,0,1,"silicon abundance interpolation type"
siabundsof,r,a,0.,0.,1.,"silicon abundance soft minimum"
siabundnst,i,a,1,1,20,"silicon abundance number of steps"
pabund,r,h,0.,0.,100.,"phosphorus abundance soft maximum"
pabundtyp,i,h,0,0,2,"phosphorus abundance variation type"
pabundint,i,h,1,0,1,"phosphorus abundance interpolation type"
pabundsof,r,h,0.,0.,1.,"phosphorus abundance soft minimum"
pabundnst,i,h,1,1,20,"phosphorus abundance number of steps"
sabund,r,h,1.,0.,100.,"sulfur abundance soft maximum"
sabundtyp,i,h,1,0,2,"sulfur abundance variation type"
sabundint,i,a,1,0,1,"sulfur abundance interpolation type"
sabundsof,r,a,0.,0.,1.,"sulfur abundance soft minimum"
sabundnst,i,a,1,1,20,"sulfur abundance number of steps"
clabund,r,h,0.,0.,100.,"chlorine abundance soft maximum"
clabundtyp,i,h,0,0,2,"chlorine abundance variation type"
clabundint,i,h,1,0,1,"chlorine abundance interpolation type"
clabundsof,r,h,0.,0.,1.,"chlorine abundance soft minimum"
clabundnst,i,h,1,1,20,"chlorine abundance number of steps"
arabund,r,h,1.,0.,100.,"argon abundance soft maximum"
arabundtyp,i,h,1,0,2,"argon abundance variation type"
arabundint,i,a,1,0,1,"argon abundance interpolation type"
arabundsof,r,a,0.,0.,1.,"argon abundance soft minimum"
arabundnst,i,a,1,1,20,"argon abundance number of steps"
kabund,r,h,0.0,0.,100.,"potassium abundance soft maximum"
kabundtyp,i,h,0,0,2,"postassium abundance variation type"
kabundint,i,h,1,0,1,"postassium abundance interpolation type"
kabundsof,r,h,0.,0.,1.,"postassium abundance soft minimum"
kabundnst,i,h,1,1,20,"postassium abundance number of steps"
caabund,r,h,1.,0.,100.,"calcium abundance soft maximum"
caabundtyp,i,h,1,0,2,"calcium abundance variation type"
caabundint,i,a,1,0,1,"calcium abundance interpolation type"
caabundsof,r,a,0.,0.,1.,"calcium abundance soft minimum"
caabundnst,i,a,1,1,20,"calcium abundance number of steps"
scabund,r,h,0.,0.,100.,"scandium abundance soft maximum"
scabundtyp,i,h,0,0,2,"scandium abundance variation type"
scabundint,i,h,1,0,1,"scandium abundance interpolation type"
scabundsof,r,h,0.,0.,1.,"scandium abundance soft minimum"
scabundnst,i,h,1,1,20,"scandium abundance number of steps"
tiabund,r,h,0.,0.,100.,"titanium abundance soft maximum"
tiabundtyp,i,h,0,0,2,"titanium abundance variation type"
tiabundint,i,h,1,0,1,"titanium abundance interpolation type"
tiabundsof,r,h,0.,0.,1.,"titanium abundance soft minimum"
tiabundnst,i,h,1,1,20,"titanium abundance number of steps"
vabund,r,h,0.,0.,100.,"vanadium abundance soft maximum"
vabundtyp,i,h,0,0,2,"vanadium abundance variation type"
vabundint,i,h,1,0,1,"vanadium abundance interpolation type"
vabundsof,r,h,0.,0.,1.,"vanadium abundance soft minimum"
vabundnst,i,h,1,1,20,"vanadium abundance number of steps"
crabund,r,h,0.,0.,100.,"chromium abundance soft maximum"
crabundtyp,i,h,0,0,2,"chromium abundance variation type"
crabundint,i,h,1,0,1,"chromium abundance interpolation type"
crabundsof,r,h,0.,0.,1.,"chromium abundance soft minimum"
crabundnst,i,h,1,1,20,"chromium abundance number of steps"
mnabund,r,h,0.,0.,100.,"manganese abundance soft maximum"
mnabundtyp,i,h,0,0,2,"manganese abundance variation type"
mnabundint,i,h,1,0,1,"manganese abundance interpolation type"
mnabundsof,r,h,0.,0.,1.,"manganese abundance soft minimum"
mnabundnst,i,h,1,1,20,"manganese abundance number of steps"
feabund,r,h,1.,0.,100.,"iron abundance soft maximum"
feabundtyp,i,h,1,0,2,"iron abundance variation type"
feabundint,i,a,1,0,1,"iron abundance interpolation type"
feabundsof,r,a,0.,0.,1.,"iron abundance soft minimum"
feabundnst,i,a,1,1,20,"iron abundance number of steps"
coabund,r,h,0.,0.,100.,"cobalt abundance soft maximum"
coabundtyp,i,h,0,0,2,"cobalt abundance variation type"
coabundint,i,h,1,0,1,"cobalt abundance interpolation type"
coabundsof,r,h,0.,0.,1.,"cobalt abundance soft minimum"
coabundnst,i,h,1,1,20,"cobalt abundance number of steps"
niabund,r,h,0.,0.,100.,"nickel abundance soft maximum"
niabundtyp,i,h,0,0,2,"nickel abundance variation type"
niabundint,i,a,1,0,1,"nickel abundance interpolation type"
niabundsof,r,a,0.,0.,1.,"nickel abundance soft minimum"
niabundnst,i,a,1,1,20,"nickel abundance number of steps"
cuabund,r,h,0.,0.,100.,"copper abundance soft maximum"
cuabundtyp,i,h,0,0,2,"copper abundance variation type"
cuabundint,i,h,1,0,1,"copper abundance interpolation type"
cuabundsof,r,h,0.,0.,1.,"copper abundance soft minimum"
cuabundnst,i,h,1,1,20,"copper abundance number of steps"
znabund,r,h,0.,0.,100.,"zinc abundance soft maximum"
znabundtyp,i,h,0,0,2,"zinc abundance variation type"
znabundint,i,h,1,0,1,"zinc abundance interpolation type"
znabundsof,r,h,0.,0.,1.,"zinc abundance soft minimum"
znabundnst,i,h,1,1,20,"zinc abundance number of steps"
spectrum,s,a,"pow",,,"spectrum type?"
spectrum_file,s,a,"spct.dat",,,"spectrum file?"
spectun,i,a,0,0,1,"spectrum units? (0=energy, 1=photons)"
redshift,i,h,1,0,1,"Is redshift a parameter? (0=no, 1=yes)"
nsteps,i,h,3,1,1000,"number of steps"
niter,i,h,0,,,"number of iterations"
lwrite,i,h,0,0,1,"write switch (1=yes, 0=no)"
lprint,i,h,0,0,1,"print switch (1=yes, 0=no)"
lstep,i,h,0,,,"step size choice switch"
npass,i,h,1,1,10000,"number of passes"
lcpres,i,h,0,0,1,"constant pressure switch (1=yes, 0=no)"
emult,r,h,1.,1.e-6,1.e+6,"Courant multiplier"
taumax,r,h,2.,1.,10000.,"tau max for courant step"
xeemin,r,h,1.e-6,1.e-6,0.5,"minimum electron fraction"
critf,r,h,1.e-14,1.e-24,0.1,"critical ion abundance"
vturbi,r,h,1.,0.,30000.,"turbulent velocity (km/s)"
radexp,r,h,0.,-3.,3.,"density distribution power law index"
ncn2,i,h,9999,999,99999,"number of continuum bins"
modelname,s,a,"coronal grid",,,"model name"
loopcontrol,i,h,0,0,30000,"loop control (0=standalone)"
elow,r,h,1.0E+2,0.,5.11E+5,"energy band low end (eV)"
ehigh,r,h,2.0E+4,0.,5.11E+5,"energy band high end (eV)"
mode,s,h,"ql",,,"mode"
\end{verbatim}


\subsection{Example 2: Photoionized Grid}

In this example a grid of photoionization models is calculated 
with varying ionization parameter, column density, and element 
abundances.  The ionizing spectrum is a power law with index -1.


\begin{verbatim}

cfrac,r,a,0.,0.,1.,"covering fraction soft maximum"
cfractyp,i,h,0,0,2,"covering fraction variation type"
cfracint,i,a,1,0,1,"covering fraction interpolation type"
cfracsof,r,a,0.,0.,1.,"covering fraction soft minimum"
cfracnst,i,a,1,1,20,"covering fraction number of steps"
temperature,r,h,1.,0.,1.E4,"temperature soft maximum (/10**4K)"
temperaturetyp,i,h,0,0,2,"temperature variation type"
temperatureint,i,a,1,0,1,"temperature interpolation type"
temperaturesof,r,a,0.,0.,1.,"temperature soft minimum"
temperaturenst,i,a,1,1,20,"temperature number of steps"
pressure,r,h,0.03,0.,1.,"pressure soft maximum (dyne/cm**2)"
pressuretyp,i,h,0,0,2,"pressure variation type"
pressureint,i,a,1,0,1,"pressure interpolation type"
pressuresof,r,a,0.,0.,1.,"pressure soft minimum"
pressurenst,i,a,1,1,20,"pressure number of steps"
density,r,a,1.E+10,0.,1.E18,"density soft maximum (cm**-3)"
densitytyp,i,h,0,0,2,"density variation type"
densityint,i,a,1,0,1,"density interpolation type"
densitysof,r,a,0.,0.,1.,"density soft minimum"
densitynst,i,a,1,1,20,"density number of steps"
trad,r,a,-1.,,,"radiation temperature or alpha soft maximum?"
tradtyp,i,h,0,0,2,"radiation temperature variation type"
tradint,i,a,1,0,1,"radiation temperature interpolation type"
tradsof,r,a,0.,0.,1.,"radiation temperature soft minimum"
tradnst,i,a,1,1,20,"radiation temperature number of steps"
rlrad38,r,a,1.e+6,0.,1.E10,"luminosity soft maximum (/10**38 erg/s)"
rlrad38typ,i,h,0,0,2,"luminosity variation type"
rlrad38int,i,a,1,0,1,"luminosity interpolation type"
rlrad38sof,r,a,0.,0.,1.,"luminosity soft minimum"
rlrad38nst,i,a,1,1,20,"luminosity number of steps"
column,r,a,1.E23,0.,1.E25,"column density soft maximum (cm**-2)"
columntyp,i,h,2,0,2,"column density variation type"
columnint,i,a,1,0,1,"column density interpolation type"
columnsof,r,a,1.E20,1.,1.E25,"column density soft minimum"
columnnst,i,a,7,1,20,"column density number of steps"
rlogxi,r,a,4.5,-10.,+10.,"log(ionization parameter) soft maximum (erg cm/s)"
rlogxityp,i,h,2,0,2,"log(ionization parameter) variation type"
rlogxiint,i,a,0,0,1,"log(ionization parameter) interpolation type"
rlogxisof,r,a,-1.,-10.0,+10.0,"log(ionization parameter) soft minimum"
rlogxinst,i,a,20,1,20,"log(ionization parameter) number of steps"
habund,r,h,1.,0.,100.,"hydrogen abundance soft maximum"
habundtyp,i,h,0,0,2,"hydrogen abundance variation type"
habundint,i,a,1,0,1,"hydrogen abundance interpolation type"
habundsof,r,a,0.,0.,1.,"hydrogen abundance soft minimum"
habundnst,i,a,1,1,20,"hydrogen abundance number of steps"
heabund,r,h,1.,0.,100.,"helium abundance soft maximum"
heabundtyp,i,h,0,0,2,"helium abundance variation type"
heabundint,i,a,1,0,1,"helium abundance interpolation type"
heabundsof,r,a,0.,0.,1.,"helium abundance soft minimum"
heabundnst,i,a,1,1,20,"helium abundance number of steps"
cabund,r,h,1.,0.,100.,"carbon abundance soft maximum"
cabundtyp,i,h,1,0,2,"carbon abundance variation type"
cabundint,i,a,1,0,1,"carbon abundance interpolation type"
cabundsof,r,a,0.,0.,1.,"carbon abundance soft minimum"
cabundnst,i,a,1,1,20,"carbon abundance number of steps"
nabund,r,h,1.,0.,100.,"nitrogen abundance soft maximum"
nabundtyp,i,h,1,0,2,"nitrogen abundance variation type"
nabundint,i,a,1,0,1,"nitrogen abundance interpolation type"
nabundsof,r,a,0.,0.,1.,"nitrogen abundance soft minimum"
nabundnst,i,a,1,1,20,"nitrogen abundance number of steps"
oabund,r,h,1.,0.,100.,"oxygen abundance soft maximum"
oabundtyp,i,h,1,0,2,"oxygen abundance variation type"
oabundint,i,a,1,0,1,"oxygen abundance interpolation type"
oabundsof,r,a,0.,0.,1.,"oxygen abundance soft minimum"
oabundnst,i,a,1,1,20,"oxygen abundance number of steps"
neabund,r,h,1.,0.,100.,"neon abundance soft maximum"
neabundtyp,i,h,1,0,2,"neon abundance variation type"
neabundint,i,a,1,0,1,"neon abundance interpolation type"
neabundsof,r,a,0.,0.,1.,"neon abundance soft minimum"
neabundnst,i,a,1,1,20,"neon abundance number of steps"
mgabund,r,h,1.,0.,100.,"magnesium abundance soft maximum"
mgabundtyp,i,h,1,0,2,"magnesium abundance variation type"
mgabundint,i,a,1,0,1,"magnesium abundance interpolation type"
mgabundsof,r,a,0.,0.,1.,"magnesium abundance soft minimum"
mgabundnst,i,a,1,1,20,"magnesium abundance number of steps"
siabund,r,h,1.,0.,100.,"silicon abundance soft maximum"
siabundtyp,i,h,1,0,2,"silicon abundance variation type"
siabundint,i,a,1,0,1,"silicon abundance interpolation type"
siabundsof,r,a,0.,0.,1.,"silicon abundance soft minimum"
siabundnst,i,a,1,1,20,"silicon abundance number of steps"
sabund,r,h,1.,0.,100.,"sulfur abundance soft maximum"
sabundtyp,i,h,1,0,2,"sulfur abundance variation type"
sabundint,i,a,1,0,1,"sulfur abundance interpolation type"
sabundsof,r,a,0.,0.,1.,"sulfur abundance soft minimum"
sabundnst,i,a,1,1,20,"sulfur abundance number of steps"
arabund,r,h,1.,0.,100.,"argon abundance soft maximum"
arabundtyp,i,h,1,0,2,"argon abundance variation type"
arabundint,i,a,1,0,1,"argon abundance interpolation type"
arabundsof,r,a,0.,0.,1.,"argon abundance soft minimum"
arabundnst,i,a,1,1,20,"argon abundance number of steps"
caabund,r,h,1.,0.,100.,"calcium abundance soft maximum"
caabundtyp,i,h,1,0,2,"calcium abundance variation type"
caabundint,i,a,1,0,1,"calcium abundance interpolation type"
caabundsof,r,a,0.,0.,1.,"calcium abundance soft minimum"
caabundnst,i,a,1,1,20,"calcium abundance number of steps"
feabund,r,h,1.,0.,100.,"iron abundance soft maximum"
feabundtyp,i,h,1,0,2,"iron abundance variation type"
feabundint,i,a,1,0,1,"iron abundance interpolation type"
feabundsof,r,a,0.,0.,1.,"iron abundance soft minimum"
feabundnst,i,a,1,1,20,"iron abundance number of steps"
niabund,r,h,0.,0.,100.,"nickel abundance soft maximum"
niabundtyp,i,h,0,0,2,"nickel abundance variation type"
niabundint,i,a,1,0,1,"nickel abundance interpolation type"
niabundsof,r,a,0.,0.,1.,"nickel abundance soft minimum"
niabundnst,i,a,1,1,20,"nickel abundance number of steps"
spectrum,s,a,"pow",,,"spectrum type?"
spectrum_file,s,a,"spct.dat",,,"spectrum file?"
spectun,i,a,0,0,1,"spectrum units? (0=energy, 1=photons)"
redshift,i,h,1,0,1,"Is redshift a parameter? (0=no, 1=yes)"
nsteps,i,h,3,1,1000,"number of steps"
niter,i,h,99,,,"number of iterations"
lwrite,i,h,0,0,1,"write switch (1=yes, 0=no)"
lprint,i,h,0,0,1,"print switch (1=yes, 0=no)"
lstep,i,h,0,,,"step size choice switch"
npass,i,h,1,1,10000,"number of passes"
lcpres,i,h,0,0,1,"constant pressure switch (1=yes, 0=no)"
emult,r,h,0.5,1.e-6,1.e+6,"Courant multiplier"
taumax,r,h,5.,1.,10000.,"tau max for courant step"
xeemin,r,h,0.1,1.e-6,0.5,"minimum electron fraction"
critf,r,h,1.e-14,1.e-24,0.1,"critical ion abundance"
vturbi,r,h,1.,0.,30000.,"turbulent velocity (km/s)"
modelname,s,a,"photoionized grid",,,"model name"
loopcontrol,i,h,0,0,30000,"loop control (0=standalone)"
elow,r,h,1.0E+2,0.,5.11E+5,"energy band low end (eV)"
ehigh,r,h,2.0E+4,0.,5.11E+5,"energy band high end (eV)"
mode,s,h,"ql",,,"mode"

\end{verbatim}


\subsection{Example 3: Photoionized Grid; Exploring the T-$\xi$ Plane}

In this example a grid of photoionization models is calculated 
each with fixed ionization parameter and temperature.  The column 
densities are all small, so that each model is effectively optically thin 
and isothermal.  The output of these models which is of interest is in the 
ascii file xstar2xspec.log, and this file can be parsed to extract quantities
such as heating and cooling rates vs. xi and T.
The ionizing spectrum is a power law with index -1.
The xstinitable.par file which produces this is:

\begin{verbatim}
cfrac,r,h,1.,0.,1.,"covering fraction soft maximum"
cfractyp,i,h,0,0,2,"covering fraction variation type"
cfracint,i,h,1,0,1,"covering fraction interpolation type"
cfracsof,r,h,0.,0.,1.,"covering fraction soft minimum"
cfracnst,i,h,1,1,20,"covering fraction number of steps"
temperature,r,h,1000.,0.3,1.E4,"temperature soft maximum (/10**4K)"
temperaturetyp,i,h,2,0,2,"temperature variation type"
temperatureint,i,h,1,0,1,"temperature interpolation type"
temperaturesof,r,h,1.,0.,1.,"temperature soft minimum"
temperaturenst,i,h,10,1,20,"temperature number of steps"
pressure,r,h,0.03,0.00000001,1.,"pressure soft maximum (dyne/cm**2)"
pressuretyp,i,h,0,0,2,"pressure variation type"
pressureint,i,h,1,0,1,"pressure interpolation type"
pressuresof,r,h,1.e-9,1.e-10,1.,"pressure soft minimum"
pressurenst,i,h,1,1,20,"pressure number of steps"
density,r,h,1.E+12,1.e+4,1.E21,"density soft maximum (cm**-3)"
densitytyp,i,h,0,0,2,"density variation type"
densityint,i,h,1,0,1,"density interpolation type"
densitysof,r,h,1e10,1.e+4,1.e+18,"density soft minimum"
densitynst,i,h,1,1,20,"density number of steps"
trad,r,h,-1.,,,"radiation temperature or alpha soft maximum?"
tradtyp,i,h,0,0,2,"radiation temperature variation type"
tradint,i,h,1,0,1,"radiation temperature interpolation type"
tradsof,r,h,0.,0.,1.,"radiation temperature soft minimum"
tradnst,i,h,1,1,20,"radiation temperature number of steps"
rlrad38,r,h,1.e+6,1.e-20,1.E10,"luminosity soft maximum (/10**38 erg/s)"
rlrad38typ,i,h,0,0,2,"luminosity variation type"
rlrad38int,i,h,1,0,1,"luminosity interpolation type"
rlrad38sof,r,h,1e1,1.e-23,1.e+10,"luminosity soft minimum"
rlrad38nst,i,h,1,1,20,"luminosity number of steps"
column,r,h,1.e+10,1.e+10,1.E25,"column density soft maximum (cm**-2)"
columntyp,i,h,0,0,2,"column density variation type"
columnint,i,h,1,0,1,"column density interpolation type"
columnsof,r,h,1.E+10,1.,1.E25,"column density soft minimum"
columnnst,i,h,1,1,20,"column density number of steps"
rlogxi,r,h,5.,-10.,+10.,"log(ionization parameter) soft maximum (erg cm/s)"
rlogxityp,i,h,2,0,2,"log(ionization parameter) variation type"
rlogxiint,i,h,0,0,1,"log(ionization parameter) interpolation type"
rlogxisof,r,h,1.,-10.0,+10.0,"log(ionization parameter) soft minimum"
rlogxinst,i,h,10,1,20,"log(ionization parameter) number of steps"
habund,r,h,1.,0.,100.,"hydrogen abundance soft maximum"
habundtyp,i,h,0,0,2,"hydrogen abundance variation type"
habundint,i,h,1,0,1,"hydrogen abundance interpolation type"
habundsof,r,h,0.,0.,1.,"hydrogen abundance soft minimum"
habundnst,i,h,1,1,20,"hydrogen abundance number of steps"
heabund,r,h,1.,0.,100.,"helium abundance soft maximum"
heabundtyp,i,h,0,0,2,"helium abundance variation type"
heabundint,i,h,1,0,1,"helium abundance interpolation type"
heabundsof,r,h,0.,0.,1.,"helium abundance soft minimum"
heabundnst,i,h,1,1,20,"helium abundance number of steps"
liabund,r,h,0.,0.,100.,"lithium abundance soft maximum"
liabundtyp,i,h,0,0,2,"lithium abundance variation type"
liabundint,i,h,1,0,1,"lithium abundance interpolation type"
liabundsof,r,h,0.,0.,1.,"lithium abundance soft minimum"
liabundnst,i,h,1,1,20,"lithium abundance number of steps"
beabund,r,h,0.,0.,100.,"beryllium abundance soft maximum"
beabundtyp,i,h,0,0,2,"beryllium abundance variation type"
beabundint,i,h,1,0,1,"beryllium abundance interpolation type"
beabundsof,r,h,0.,0.,1.,"beryllium abundance soft minimum"
beabundnst,i,h,1,1,20,"beryllium abundance number of steps"
babund,r,h,0.,0.,100.,"boron abundance soft maximum"
babundtyp,i,h,0,0,2,"boron abundance variation type"
babundint,i,h,1,0,1,"boron abundance interpolation type"
babundsof,r,h,0.,0.,1.,"boron abundance soft minimum"
babundnst,i,h,1,1,20,"boron abundance number of steps"
cabund,r,h,1.,0.,100.,"carbon abundance soft maximum"
cabundtyp,i,h,0,0,2,"carbon abundance variation type"
cabundint,i,h,1,0,1,"carbon abundance interpolation type"
cabundsof,r,h,0.,0.,1.,"carbon abundance soft minimum"
cabundnst,i,h,1,1,20,"carbon abundance number of steps"
nabund,r,h,1.,0.,100.,"nitrogen abundance soft maximum"
nabundtyp,i,h,0,0,2,"nitrogen abundance variation type"
nabundint,i,h,1,0,1,"nitrogen abundance interpolation type"
nabundsof,r,h,0.,0.,1.,"nitrogen abundance soft minimum"
nabundnst,i,h,1,1,20,"nitrogen abundance number of steps"
oabund,r,h,1.,0.,100.,"oxygen abundance soft maximum"
oabundtyp,i,h,0,0,2,"oxygen abundance variation type"
oabundint,i,h,1,0,1,"oxygen abundance interpolation type"
oabundsof,r,h,0.,0.,1.,"oxygen abundance soft minimum"
oabundnst,i,h,1,1,20,"oxygen abundance number of steps"
fabund,r,h,0.,0.,100.,"fluorine abundance soft maximum"
fabundtyp,i,h,0,0,2,"fluorine abundance variation type"
fabundint,i,h,1,0,1,"fluorine abundance interpolation type"
fabundsof,r,h,0.,0.,1.,"fluorine abundance soft minimum"
fabundnst,i,h,1,1,20,"fluorine abundance number of steps"
neabund,r,h,1.,0.,100.,"neon abundance soft maximum"
neabundtyp,i,h,0,0,2,"neon abundance variation type"
neabundint,i,h,1,0,1,"neon abundance interpolation type"
neabundsof,r,h,0.,0.,1.,"neon abundance soft minimum"
neabundnst,i,h,1,1,20,"neon abundance number of steps"
naabund,r,h,0.,0.,100.,"sodium abundance soft maximum"
naabundtyp,i,h,0,0,2,"sodium abundance variation type"
naabundint,i,h,1,0,1,"sodium abundance interpolation type"
naabundsof,r,h,0.,0.,1.,"sodium abundance soft minimum"
naabundnst,i,h,1,1,20,"sodium abundance number of steps"
mgabund,r,h,1.,0.,100.,"magnesium abundance soft maximum"
mgabundtyp,i,h,0,0,2,"magnesium abundance variation type"
mgabundint,i,h,1,0,1,"magnesium abundance interpolation type"
mgabundsof,r,h,0.,0.,1.,"magnesium abundance soft minimum"
mgabundnst,i,h,1,1,20,"magnesium abundance number of steps"
alabund,r,h,0.,0.,100.,"aluminium abundance soft maximum"
alabundtyp,i,h,0,0,2,"aluminium abundance variation type"
alabundint,i,h,1,0,1,"aluminium abundance interpolation type"
alabundsof,r,h,0.,0.,1.,"aluminium abundance soft minimum"
alabundnst,i,h,1,1,20,"aluminium abundance number of steps"
siabund,r,h,1.,0.,100.,"silicon abundance soft maximum"
siabundtyp,i,h,0,0,2,"silicon abundance variation type"
siabundint,i,h,1,0,1,"silicon abundance interpolation type"
siabundsof,r,h,0.,0.,1.,"silicon abundance soft minimum"
siabundnst,i,h,1,1,20,"silicon abundance number of steps"
pabund,r,h,0.,0.,100.,"phosphorus abundance soft maximum"
pabundtyp,i,h,0,0,2,"phosphorus abundance variation type"
pabundint,i,h,1,0,1,"phosphorus abundance interpolation type"
pabundsof,r,h,0.,0.,1.,"phosphorus abundance soft minimum"
pabundnst,i,h,1,1,20,"phosphorus abundance number of steps"
sabund,r,h,1.,0.,100.,"sulfur abundance soft maximum"
sabundtyp,i,h,0,0,2,"sulfur abundance variation type"
sabundint,i,h,1,0,1,"sulfur abundance interpolation type"
sabundsof,r,h,0.,0.,1.,"sulfur abundance soft minimum"
sabundnst,i,h,1,1,20,"sulfur abundance number of steps"
clabund,r,h,0.,0.,100.,"chlorine abundance soft maximum"
clabundtyp,i,h,0,0,2,"chlorine abundance variation type"
clabundint,i,h,1,0,1,"chlorine abundance interpolation type"
clabundsof,r,h,0.,0.,1.,"chlorine abundance soft minimum"
clabundnst,i,h,1,1,20,"chlorine abundance number of steps"
arabund,r,h,1.,0.,100.,"argon abundance soft maximum"
arabundtyp,i,h,0,0,2,"argon abundance variation type"
arabundint,i,h,1,0,1,"argon abundance interpolation type"
arabundsof,r,h,0.,0.,1.,"argon abundance soft minimum"
arabundnst,i,h,1,1,20,"argon abundance number of steps"
kabund,r,h,0.0,0.,100.,"potassium abundance soft maximum"
kabundtyp,i,h,0,0,2,"postassium abundance variation type"
kabundint,i,h,1,0,1,"postassium abundance interpolation type"
kabundsof,r,h,0.,0.,1.,"postassium abundance soft minimum"
kabundnst,i,h,1,1,20,"postassium abundance number of steps"
caabund,r,h,1.,0.,100.,"calcium abundance soft maximum"
caabundtyp,i,h,0,0,2,"calcium abundance variation type"
caabundint,i,h,1,0,1,"calcium abundance interpolation type"
caabundsof,r,h,0.,0.,1.,"calcium abundance soft minimum"
caabundnst,i,h,1,1,20,"calcium abundance number of steps"
scabund,r,h,0.,0.,100.,"scandium abundance soft maximum"
scabundtyp,i,h,0,0,2,"scandium abundance variation type"
scabundint,i,h,1,0,1,"scandium abundance interpolation type"
scabundsof,r,h,0.,0.,1.,"scandium abundance soft minimum"
scabundnst,i,h,1,1,20,"scandium abundance number of steps"
tiabund,r,h,0.,0.,100.,"titanium abundance soft maximum"
tiabundtyp,i,h,0,0,2,"titanium abundance variation type"
tiabundint,i,h,1,0,1,"titanium abundance interpolation type"
tiabundsof,r,h,0.,0.,1.,"titanium abundance soft minimum"
tiabundnst,i,h,1,1,20,"titanium abundance number of steps"
vabund,r,h,0.,0.,100.,"vanadium abundance soft maximum"
vabundtyp,i,h,0,0,2,"vanadium abundance variation type"
vabundint,i,h,1,0,1,"vanadium abundance interpolation type"
vabundsof,r,h,0.,0.,1.,"vanadium abundance soft minimum"
vabundnst,i,h,1,1,20,"vanadium abundance number of steps"
crabund,r,h,0.,0.,100.,"chromium abundance soft maximum"
crabundtyp,i,h,0,0,2,"chromium abundance variation type"
crabundint,i,h,1,0,1,"chromium abundance interpolation type"
crabundsof,r,h,0.,0.,1.,"chromium abundance soft minimum"
crabundnst,i,h,1,1,20,"chromium abundance number of steps"
mnabund,r,h,0.,0.,100.,"manganese abundance soft maximum"
mnabundtyp,i,h,0,0,2,"manganese abundance variation type"
mnabundint,i,h,1,0,1,"manganese abundance interpolation type"
mnabundsof,r,h,0.,0.,1.,"manganese abundance soft minimum"
mnabundnst,i,h,1,1,20,"manganese abundance number of steps"
feabund,r,h,1.,0.,100.,"iron abundance soft maximum"
feabundtyp,i,h,0,0,2,"iron abundance variation type"
feabundint,i,h,1,0,1,"iron abundance interpolation type"
feabundsof,r,h,0.1,0.,1.,"iron abundance soft minimum"
feabundnst,i,h,0,1,20,"iron abundance number of steps"
coabund,r,h,0.,0.,100.,"cobalt abundance soft maximum"
coabundtyp,i,h,0,0,2,"cobalt abundance variation type"
coabundint,i,h,1,0,1,"cobalt abundance interpolation type"
coabundsof,r,h,0.,0.,1.,"cobalt abundance soft minimum"
coabundnst,i,h,1,1,20,"cobalt abundance number of steps"
niabund,r,h,0.,0.,100.,"nickel abundance soft maximum"
niabundtyp,i,h,0,0,2,"nickel abundance variation type"
niabundint,i,h,1,0,1,"nickel abundance interpolation type"
niabundsof,r,h,0.,0.,1.,"nickel abundance soft minimum"
niabundnst,i,h,1,1,20,"nickel abundance number of steps"
cuabund,r,h,0.,0.,100.,"copper abundance soft maximum"
cuabundtyp,i,h,0,0,2,"copper abundance variation type"
cuabundint,i,h,1,0,1,"copper abundance interpolation type"
cuabundsof,r,h,0.,0.,1.,"copper abundance soft minimum"
cuabundnst,i,h,1,1,20,"copper abundance number of steps"
znabund,r,h,0.,0.,100.,"zinc abundance soft maximum"
znabundtyp,i,h,0,0,2,"zinc abundance variation type"
znabundint,i,h,1,0,1,"zinc abundance interpolation type"
znabundsof,r,h,0.,0.,1.,"zinc abundance soft minimum"
znabundnst,i,h,1,1,20,"zinc abundance number of steps"
spectrum,s,h,"pow",,,"spectrum type?"
spectrum_file,s,h,"bknpw5",,,"spectrum file?"
spectun,i,h,0,0,1,"spectrum units? (0=energy, 1=photons)"
redshift,i,h,1,0,1,"Is redshift a parameter? (0=no, 1=yes)"
nsteps,i,h,3,1,1000,"number of steps"
niter,i,h,0,,,"number of iterations"
lwrite,i,h,0,0,1,"write switch (1=yes, 0=no)"
lprint,i,h,0,0,1,"print switch (1=yes, 0=no)"
lstep,i,h,0,,,"step size choice switch"
npass,i,h,1,1,10000,"number of passes"
lcpres,i,h,0,0,1,"constant pressure switch (1=yes, 0=no)"
emult,r,h,0.5,1.e-6,1.e+6,"Courant multiplier"
taumax,r,h,5.,1.,10000.,"tau max for courant step"
xeemin,r,h,0.1,1.e-6,0.5,"minimum electron fraction"
critf,r,h,1.e-4,1.e-24,0.1,"critical ion abundance"
vturbi,r,h,300.,0.,30000.,"turbulent velocity (km/s)"
radexp,r,h,0.,-3.,3.,"density distribution power law index"
ncn2,i,h,999,999,99999,"number of continuum bins"
modelname,s,h,"template",,,"model name"
loopcontrol,i,h,0,0,30000,"loop control (0=standalone)"
elow,r,h,1.0E+2,0.,5.11E+5,"energy band low end (eV)"
ehigh,r,h,2.0E+4,0.,5.11E+5,"energy band high end (eV)"
mode,s,h,"ql",,,"mode"

\end{verbatim}

A very simple Python script which parses this file and prints out the values of 
log(xi), T, httot, cltot, is:

\begin{verbatim}

#!/usr/bin/python
import sys
WANTED=11
a=[]
with open('xstar2xspec.log', 'r') as inF:
  for line in inF:
    left,sep,right = line.partition(' log(xi)=')
    if sep: 
       atmp=float(right[:WANTED])
       a.append(atmp)
    left,sep,right = line.partition(' log(xi)=')
    if sep: 
       atmp=float(right[:WANTED])
       a.append(atmp)
    left,sep,right = line.partition('httot=')
    if sep: 
       atmp=float(right[:WANTED])
       a.append(atmp)
    left,sep,right = line.partition('cltot=')
    if sep: 
       atmp=float(right[:WANTED])
       a.append(atmp)
       print " ".join('%0.2f' % item for item in a)
       a=[]

\end{verbatim}

This can be used to make figures such as this:

\begin{figure}
\epsfxsize=5.6in  % narrow the plot
\epsfysize=7.0in  % shorten the plot
\epsffile{f70a.ps}
\caption{Figure showing contours of constant heating - cooling in the 
(T-$\xi$) plane for an illuminating spectrum which is a $\gamma$=2 power law.  
Equlibrium is shown as the solid curve.  Figures 
such as this can be created using the input in this section.}
\end{figure}

