\documentclass[11pt]{book}
\input{html.sty}
\htmladdtonavigation
   {\begin{rawhtml}
 <A HREF="http://heasarc.gsfc.nasa.gov/docs/xanadu/xspec">XSPEC Home</A>
    \end{rawhtml}}
\oddsidemargin=0.00in
\evensidemargin=0.00in
\textwidth=6.5in
\topmargin=0.0in
\textheight=8.5in
\parindent=0cm
\parskip=0.2cm
\begin{document}

\begin{titlepage}
\normalsize
\vspace*{4.0cm}
\begin{center}
{\Huge \bf XSFunctions Guide}\\
\end{center}
\medskip
\medskip
\begin{center}
{\Large Version 1.01 \\}
\end{center}
\bigskip
\begin{center}
{\Large Keith A. Arnaud \\}
\end{center}
\medskip
\medskip
\begin{center}
{HEASARC\\
Code 662\\
Goddard Space Flight Center\\
Greenbelt, MD 20771\\
USA}
\end{center}

\vfill
\bigskip
\begin{center}
{\Large Oct 2014\\}
\end{center}
\vfill
\end{titlepage}

\pagenumbering{roman}

\tableofcontents
\pagenumbering{arabic}
\chapter{Introduction}

This document describes the C++ classes available in the XSPEC model
functions library (XSFunctions). While these are not controlled as
rigorously as XSPEC external interfaces, we do try to keep them as
stable as possible so we encourage their use in other XSPEC models.

The classes are
\begin{itemize}

\item Aped - generates CEI and NEI spectra using AtomDB input files.

\item IonBalNei - calculates NEI ionization balances.

\item NeutralOpacity - calculates opacities for neutral material.

\item IonizedOpacity - calculates opacities for photo-ionized material.

\item MZCompRefl - calculates Compton reflection using the Magzdiarz
  and Zdziarski code.

\end{itemize}


\chapter{Aped}

The Aped classes store and use the data stored in the AtomDB
files. The top-level class, Aped, comprises a vector of
ApedTemperatureRecord objects, each of which comprises a vector of
ApedElementRecord objects, each of which comprises a vector of
ApedIonRecord objects. Thus, there is one ApedIonRecord for
each temperature, element, and ion in the AtomDB files. The top-level Aped
class provides the methods to load and use the atomic data.

\section{ApedIonRecord class}

\begin{verbatim}
class ApedIonRecord{
 public:

  int Ion;
  RealArray ContinuumEnergy;
  RealArray ContinuumFlux;
  RealArray ContinuumFluxError;
  RealArray PseudoContinuumEnergy;
  RealArray PseudoContinuumFlux;
  RealArray PseudoContinuumFluxError;
  RealArray LineEnergy;
  RealArray LineEnergyError;
  RealArray LineEmissivity;
  RealArray LineEmissivityError;
  IntegerArray ElementDriver;
  IntegerArray IonDriver;
  IntegerArray UpperLevel;
  IntegerArray LowerLevel;

  ApedIonRecord();    // default constructor
  ~ApedIonRecord();   // destructor

  ApedIonRecord& operator=(const ApedIonRecord&);    // deep copy

};
\end{verbatim}

The class includes entries to store error estimates on the fluxes and
emissivities although these are not used at the time of writing. Note
that the same classes are used for NEI and CEI although for the latter the
ElementDriver and IonDriver arrays are not required.

\section{ApedElementRecord class}

\begin{verbatim}
class ApedElementRecord{
 public:

  int AtomicNumber;
  vector<ApedIonRecord> IonRecord;

  ApedElementRecord();    // default constructor
  ~ApedElementRecord();   // destructor

  void LoadIonRecord(ApedIonRecord input);

  ApedElementRecord& operator=(const ApedElementRecord&);    // deep copy

};
\end{verbatim}

Each ApedElementRecord object stores the atomic number of the element
and the records for each of its ions.

\section{ApedTemperatureRecord}

\begin{verbatim}
class ApedTemperatureRecord{
 public:

  Real Temperature;
  vector<ApedElementRecord> ElementRecord;

  ApedTemperatureRecord();     //default constructor
  ~ApedTemperatureRecord();    //destructor

  void LoadElementRecord(ApedElementRecord input);

  ApedTemperatureRecord& operator=(const ApedTemperatureRecord&);    // deep copy

};
\end{verbatim}

Each ApedTemperatureRecord object stores the temperature and the
records for each element.

\section{Aped class}

\begin{verbatim}
class Aped{
 public:

  vector<ApedTemperatureRecord> TemperatureRecord;

  // These store the information in the initial PARAMETERS extension

  RealArray Temperatures;
  IntegerArray NelementsLine;
  IntegerArray NelementsCoco;
  IntegerArray Nline;
  IntegerArray Ncont;

  string coconame;
  string linename;

  bool noLines;
  bool thermalBroadening;
  Real velocityBroadening;

  Aped();     //default constructor
  ~Aped();    //destructor

  Aped& operator=(const Aped&);    // deep copy
\end{verbatim}

The class provides methods to load the data from the
AtomDB continuum and line files. The Read method just gets
the information from the PARAMETERS extension, stores the filenames
used and sizes the subsidiary objects correctly. To actually load the 
data use ReadTemperature. 

\begin{verbatim}
  // Reads the continuum and line files and stores the names and
  // temperatures. Does not read the actual continuum and line data.

  int Read(string cocofilename, string linefilename);

  int ReadTemperature(const int TemperatureIndex);
  int ReadTemperature(const vector<int>& TemperatureIndex);
\end{verbatim}

The noLines, thermalBroadening and velocityBroadening elements can be
set using the following methods. Each of these methods also checks for
the corresponding xset variables: APECNOLINES, APECTHERMAL, and
APECVELOCITY, respectively. For the first two the xset variable
overrides the value passed to the method while for the velocity case a
warning will be issued if the input value is non-zero and differs from
APECVELOCITY. If noLines is set then the spectrum will only be
continuum. If thermalBroadening is set then any lines will be
thermally broadened while if velocityBroadening is set then any lines
will be broadened by the appropriate velocity. The line shape is
assumed Gaussian and thermal and velocity broadening are added in
quadrature. Note that line broadened spectra take longer to calculate.

\begin{verbatim}
  void SetNoLines(const bool qno);
  void SetThermalBroadening(const bool qtherm);
  void SetVelocityBroadening(const Real velocity);
\end{verbatim}

The following provide methods to extract some useful numbers.

\begin{verbatim}
  int NumberTemperatures();     // return number of tabulated temperatures
  int NumberElements();         // return number of tabulated elements
  int NumberIons(int Z);        // return number of ions across all temperatures
                                // for element Z.
\end{verbatim}

Whether the data for a given temperature index has been loaded can be
checked using IsTemperatureLoaded.

\begin{verbatim}
 bool IsTemperatureLoaded(const int TemperatureIndex);
\end{verbatim}

The method to generate spectra for CEI plasmas is SumEqSpectra. This
comes in several overloaded options depending on whether a single
temperature of distribution of temperatures are required. It is also
possible to use a different temperature for thermal broadening of
lines than that used for the ionization balance. The energy bins on
which the spectrum is to be calculated are specified by standard XSPEC
energyArray. The elements required and their abundances are given by
Zinput and abundance. Tinput and Dem give the temperature(s) and
emission measure(s) required.

\begin{verbatim}
  void SumEqSpectra(const RealArray& energyArray, 
                    const IntegerArray& Zinput, const RealArray& abundance,
                    const Real Redshift, const Real& Tinput,
                    const Real& Dem, RealArray& fluxArray, 
                    RealArray& fluxErrArray);

  void SumEqSpectra(const RealArray& energyArray, 
                    const IntegerArray& Zinput, const RealArray& abundance,
                    const Real Redshift, const RealArray& Tinput,
                    const RealArray& Dem, RealArray& fluxArray, 
                    RealArray& fluxErrArray);

  // case where the temperature used for the thermal broadening differs from that
  // for the ionization

  void SumEqSpectra(const RealArray& energyArray, 
                    const IntegerArray& Zinput, const RealArray& abundance,
                    const Real Redshift, const Real& Tinput, 
                    const Real& Tbinput, const Real& Dem,
                    RealArray& fluxArray, RealArray& fluxErrArray);

  void SumEqSpectra(const RealArray& energyArray, 
                    const IntegerArray& Zinput, const RealArray& abundance,
                    const Real Redshift, const RealArray& Tinput, 
                    const RealArray& Tbinput, const RealArray& Dem, 
                    RealArray& fluxArray, RealArray& fluxErrArray);
\end{verbatim}

For NEI plasmas, the method to calculate the spectrum is
SumNeqSpectra. The inputs differ from SumEqSpectra only in including
IonFrac, which specifies the ionization fractions required for each
element, for each temperature.

\begin{verbatim}
  void SumNeqSpectra(const RealArray& energyArray, 
                     const IntegerArray& Zinput, const RealArray& abundance,
                     const Real Redshift, const Real& Tinput,
                     const vector<RealArray>& IonFrac, 
                     RealArray& fluxArray, RealArray& fluxErrArray);

  void SumNeqSpectra(const RealArray& energyArray, 
                     const IntegerArray& Zinput, const RealArray& abundance,
                     const Real Redshift, const RealArray& Tinput,
                     const vector<vector<RealArray> >& IonFrac, 
                     RealArray& fluxArray, RealArray& fluxErrArray);

  // case where the temperature used for the thermal broadening differs from that
  // for the ionization

  void SumNeqSpectra(const RealArray& energyArray, 
                     const IntegerArray& Zinput, const RealArray& abundance,
                     const Real Redshift, const Real& Tinput,
                     const Real& Tbinput, 
                     const vector<RealArray>& IonFrac, 
                     RealArray& fluxArray, RealArray& fluxErrArray);

  void SumNeqSpectra(const RealArray& energyArray, 
                     const IntegerArray& Zinput, const RealArray& abundance,
                     const Real Redshift, const RealArray& Tinput,
                     const RealArray& Tbinput, 
                     const vector<vector<RealArray> >& IonFrac, 
                     RealArray& fluxArray, RealArray& fluxErrArray);
\end{verbatim}

All versions of SumEqSpectra and SumNeqSpectra operate through

\begin{verbatim}
  void SumSpectra(const RealArray& energyArray, 
                  const IntegerArray& Zinput, const RealArray& abundance,
                  const Real Redshift, const RealArray& Tinput,
                  const RealArray& Tbinput, 
                  const vector<vector<RealArray> >& IonFrac, const bool isCEI,
                  RealArray& fluxArray, RealArray& fluxErrArray);
\end{verbatim}

Outside the class there are wrap-up routines which create an internal,
static Aped object then call SumSpectra to calculate the output
spectrum. There are overloaded versions corresponding to the options
for SumEqSpectra and SumNeqSpectra. The int returned will be non-zero
in the event of an error on reading the AtomDB files. For the CEI case
these routines check the APECROOT variable to find the files to
read. For the NEI case they check the NEIVERS and NEIAPECROOT 
variables. The calcRSSpectrum routines are for the old Raymond-Smith
model and read the RS files in the AtomDB format.

\begin{verbatim}
int calcCEISpectrum(const RealArray& energyArray, 
                    const IntegerArray& Zinput, const RealArray& abundance,
                    const Real Redshift, const Real& Tinput,
                    const Real& Dem, const bool qtherm, const Real velocity,
                    RealArray& fluxArray, RealArray& fluxErrArray);

int calcCEISpectrum(const RealArray& energyArray, 
                    const IntegerArray& Zinput, const RealArray& abundance,
                    const Real Redshift, const RealArray& Tinput,
                    const RealArray& Dem, const bool qtherm, const Real velocity,
                    RealArray& fluxArray, RealArray& fluxErrArray);

int calcCEISpectrum(const RealArray& energyArray, 
                    const IntegerArray& Zinput, const RealArray& abundance,
                    const Real Redshift, const Real& Tinput,
                    const Real& Tbinput, const Real& Dem, 
                    const bool qtherm, const Real velocity,
                    RealArray& fluxArray, RealArray& fluxErrArray);

int calcCEISpectrum(const RealArray& energyArray, 
                    const IntegerArray& Zinput, const RealArray& abundance,
                    const Real Redshift, const RealArray& Tinput,
                    const RealArray& Tbinput, const RealArray& Dem, 
                    const bool qtherm, const Real velocity,
                    RealArray& fluxArray, RealArray& fluxErrArray);

int calcCEISpectrum(const RealArray& energyArray, 
                    const IntegerArray& Zinput, const RealArray& abundance,
                    const Real Redshift, const Real& Tinput,
                    const Real& Dem, const bool qtherm, const Real velocity,
                    const bool noLines,
                    RealArray& fluxArray, RealArray& fluxErrArray);

int calcCEISpectrum(const RealArray& energyArray, 
                    const IntegerArray& Zinput, const RealArray& abundance,
                    const Real Redshift, const RealArray& Tinput,
                    const RealArray& Dem, const bool qtherm, const Real velocity,
                    const bool noLines,
                    RealArray& fluxArray, RealArray& fluxErrArray);

int calcCEISpectrum(const RealArray& energyArray, 
                    const IntegerArray& Zinput, const RealArray& abundance,
                    const Real Redshift, const Real& Tinput,
                    const Real& Tbinput, const Real& Dem, 
                    const bool qtherm, const Real velocity,
                    const bool noLines,
                    RealArray& fluxArray, RealArray& fluxErrArray);

int calcCEISpectrum(const RealArray& energyArray, 
                    const IntegerArray& Zinput, const RealArray& abundance,
                    const Real Redshift, const RealArray& Tinput,
                    const RealArray& Tbinput, const RealArray& Dem, 
                    const bool qtherm, const Real velocity,
                    const bool noLines,
                    RealArray& fluxArray, RealArray& fluxErrArray);

int calcRSSpectrum(const RealArray& energyArray, 
                   const IntegerArray& Zinput, const RealArray& abundance,
                   const Real Redshift, const Real& Tinput,
                   const Real& Dem, const bool qtherm, const Real velocity,
                   RealArray& fluxArray, RealArray& fluxErrArray);

int calcRSSpectrum(const RealArray& energyArray, 
                   const IntegerArray& Zinput, const RealArray& abundance,
                   const Real Redshift, const RealArray& Tinput,
                   const RealArray& Dem, const bool qtherm, const Real velocity,
                   RealArray& fluxArray, RealArray& fluxErrArray);

int calcRSSpectrum(const RealArray& energyArray, 
                   const IntegerArray& Zinput, const RealArray& abundance,
                   const Real Redshift, const Real& Tinput,
                   const Real& Tbinput, const Real& Dem, 
                   const bool qtherm, const Real velocity,
                   RealArray& fluxArray, RealArray& fluxErrArray);

int calcRSSpectrum(const RealArray& energyArray, 
                   const IntegerArray& Zinput, const RealArray& abundance,
                   const Real Redshift, const RealArray& Tinput,
                   const RealArray& Tbinput, const RealArray& Dem, 
                   const bool qtherm, const Real velocity,
                   RealArray& fluxArray, RealArray& fluxErrArray);
int calcNEISpectrum(const RealArray& energyArray, 
                    const IntegerArray& Zinput, const RealArray& abundance,
                    const Real Redshift, const Real& Tinput,
                    const vector<RealArray>& IonFrac, 
                    const bool qtherm, const Real velocity,
                    RealArray& fluxArray, RealArray& fluxErrArray);

int calcNEISpectrum(const RealArray& energyArray, 
                    const IntegerArray& Zinput, const RealArray& abundance,
                    const Real Redshift, const RealArray& Tinput,
                    const vector<vector<RealArray> >& IonFrac, 
                    const bool qtherm, const Real velocity,
                    RealArray& fluxArray, RealArray& fluxErrArray);

int calcNEISpectrum(const RealArray& energyArray, 
                    const IntegerArray& Zinput, const RealArray& abundance,
                    const Real Redshift, const Real& Tinput,
                    const Real& Tbinput,
                    const vector<RealArray>& IonFrac, 
                    const bool qtherm, const Real velocity,
                    RealArray& fluxArray, RealArray& fluxErrArray);

int calcNEISpectrum(const RealArray& energyArray, 
                    const IntegerArray& Zinput, const RealArray& abundance,
                    const Real Redshift, const RealArray& Tinput,
                    const RealArray& Tbinput,
                    const vector<vector<RealArray> >& IonFrac, 
                    const bool qtherm, const Real velocity,
                    RealArray& fluxArray, RealArray& fluxErrArray);
\end{verbatim}

A couple of other useful routines also live in Aped.h. getAtomicMass
returns the atomic mass for the given atomic number. apedInterpFlux is
used to interpolate the continuum and pseudo-continuum arrays.

\begin{verbatim}
Real getAtomicMass(const int& AtomicNumber);

void apedInterpFlux(const RealArray& inputEnergy, const RealArray& inputFlux, 
                    const Real& z15, const Real& coeff,
                    const RealArray& energyArray, RealArray&
fluxArray);
\end{verbatim}

\section{Examples}

An example use of the Aped class to calculate a CEI spectrum can be
found in the file smdem2.cxx in Xspec/src/XSFunctions. An example use
to calculate an NEI spectrum can be found in vvgnei.cxx in the same directory.

\chapter{IonBalNei}

The IonBalNei classes store the eigenvector data and calculate
ionization fractions for an NEI collisional plasma. The top-level
class IonBalNei comprises a vector of IonBalTemperatureRecord classes,
each of which comprises a vector of IonBalElementRecord classes. Thus
there is one IonBalElementRecord for each temperature and element in the
eigevector data files. The top-level IonBalNei class provides methods
to load and use the eigenvector data.

\section{IonBalElementRecord}

\begin{verbatim}
class IonBalElementRecord{
 public:

  int AtomicNumber;
  RealArray EquilibriumPopulation;
  RealArray Eigenvalues;
  vector<RealArray> LeftEigenvectors;
  vector<RealArray> RightEigenvectors;
  
  IonBalElementRecord();     // default constructor
  ~IonBalElementRecord();    // destructor

  void Clear();  // clear out the contents

  IonBalElementRecord& operator=(const IonBalElementRecord&);  // deep copy

};
\end{verbatim}

This class stores the eigenvector data for the element with specified
AtomicNumber. EquilibriumPopulation stores the CEI ion fractions.

\section{IonBalTemperatureRecord}

\begin{verbatim}
class IonBalTemperatureRecord{
 public:

  Real Temperature;
  vector<IonBalElementRecord> ElementRecord;

  IonBalTemperatureRecord();        // default constructor
  ~IonBalTemperatureRecord();       // destructor

  void LoadElementRecord(IonBalElementRecord input);

  void Clear();   // clear out the contents

  IonBalTemperatureRecord& operator=(const IonBalTemperatureRecord&);

};
\end{verbatim}

Each IonBalTemperatureRecord object stores the temperature and the
eigenvector data for each element.

\section{IonBalNei}

\begin{verbatim}
class IonBalNei{
 public:

  vector<IonBalTemperatureRecord> TemperatureRecord;
  RealArray Temperature;
  string Version;

  IonBalNei();       // default constructor
  ~IonBalNei();      // destructor

  IonBalNei& operator=(const IonBalNei&);    // deep copy

  void Clear();  // clear out the contents
\end{verbatim}

The Version string is used to control which version of the eigenvector
data files are to be used. If the version is changed then the object
is reset using Clear and setVersion returns true. The setVersion
method with no input checks NEIVERS and uses that if it has been set.

\begin{verbatim}
  string getVersion();
  bool setVersion();
  bool setVersion(string version);
\end{verbatim}

This class provides a Read method to load data from the eigenvector
files. This can either be done using a vector of atomic numbers and
the directory name and version string, so that the relevant filenames
can be constructed, or using a single atomic number and filename.

\begin{verbatim}
  //  Read loads eigenvector data for a set of elements

  int Read(vector<int> Z, string dirname, string versionname);

  //  ReadElement loads eigenvector data for that element

  int ReadElement(int Z, string filename);

  void LoadTemperatureRecord(IonBalTemperatureRecord input);
\end{verbatim}

The following provide methods to extract some useful numbers.

\begin{verbatim}
  RealArray Temperatures();     // return tabulated temperatures
  int NumberTemperatures();     // return number of tabulated temperatures
  int NumberElements();         // return number of tabulated elements
  bool ContainsElement(const int& Z); // returns true if data for Z
\end{verbatim}

The CEI method returns the ionization fractions for the specified
element for a collisional ionization equilibrium plasma with the given
electron temperature.

\begin{verbatim}
  RealArray CEI(const Real& Te, const int& Z);
\end{verbatim}

The method to calculate ion fractions is Calc. It is overloaded to
give a number of different options. If the initIonFrac array is used
then this gives the initial ion fractions for the element Z, if not
the element is assumed to start un-ionized. 

\begin{verbatim}
  // calculate the NEI ion fractions for electron temperature Te, ionization
  // parameter tau and element Z.

  RealArray Calc(const Real& Te, const Real& tau, const int& Z);
  RealArray Calc(const Real& Te, const Real& tau, const int& Z, 
                 const RealArray& initIonFrac);

  // Calculates ionization fractions at electron temperatures
  // Te and a set of ionization parameters tau(i), i=1,..,n,
  // where each tau is given weight(i). Electron temperature is
  // assumed to be linear function of tau.
  // Based on the old noneq.f.

  RealArray Calc(const RealArray& Te, const RealArray& tau, 
		 const RealArray& weight, const int& Z);
  RealArray Calc(const RealArray& Te, const RealArray& tau, 
		 const RealArray& weight, const int& Z, 
                 const RealArray& initIonFrac);

  // Calculates ionization fractions at electron temperature
  // Te(n) and ionization parameter tau(n), for electron 
  // temperatures Te given in a tabular form as a function of 
  // ionization parameter tau.
  // Based on the old noneqr.f.
  // Does initIonFrac make sense in this case.

  RealArray Calc(const RealArray& Te, const RealArray& tau, 
		 const int& Z);
  RealArray Calc(const RealArray& Te, const RealArray& tau, 
		 const int& Z, const RealArray& initIonFrac);

};
\end{verbatim}

Outside the class there are wrap-up functions to read (if necessary)
the eigenvector files and calculate ion fractions. They use the
NEISPEC xset variable to choose which version of the files to
use. The various overloaded versions match to the Calc methods.

\begin{verbatim}
void calcNEIfractions(const Real& Te, const Real& tau, const int& Z, 
		      RealArray& IonFrac);
void calcNEIfractions(const Real& Te, const Real& tau, const IntegerArray& Z, 
		      vector<RealArray>& IonFrac);

void calcNEIfractions(const Real& Te, const Real& tau, const int& Z, 
		      const RealArray& initIonFrac, RealArray& IonFrac);
void calcNEIfractions(const Real& Te, const Real& tau, const IntegerArray& Z, 
		      const vector<RealArray>& initIonFrac, 
		      vector<RealArray>& IonFrac);

void calcNEIfractions(const RealArray& Te, const RealArray& tau, 
		      const RealArray& weight, const int& Z, RealArray& IonFrac);
void calcNEIfractions(const RealArray& Te, const RealArray& tau, 
		      const RealArray& weight, const IntegerArray& Z, 
		      vector<RealArray>& IonFrac);

void calcNEIfractions(const RealArray& Te, const RealArray& tau, 
		      const RealArray& weight, const int& Z, 
		      const RealArray& initIonFrac, RealArray& IonFrac);
void calcNEIfractions(const RealArray& Te, const RealArray& tau, 
		      const RealArray& weight, const IntegerArray& Z, 
		      const vector<RealArray>& initIonFrac, 
		      vector<RealArray>& IonFrac);

void calcNEIfractions(const RealArray& Te, const RealArray& tau, 
		      const int& Z, RealArray& IonFrac);
void calcNEIfractions(const RealArray& Te, const RealArray& tau, 
		      const IntegerArray& Z, vector<RealArray>& IonFrac);

void calcNEIfractions(const RealArray& Te, const RealArray& tau, 
		      const int& Z, const RealArray& initIonFrac, 
		      RealArray& IonFrac);
void calcNEIfractions(const RealArray& Te, const RealArray& tau, 
		      const IntegerArray& Z, const RealArray& initIonFrac, 
		      vector<RealArray>& IonFrac);
\end{verbatim}

The wrapper routine to return the collisional ionization equilibrium
fractions is

\begin{verbatim}
void calcCEIfractions(const Real Te, const IntegerArray& Z, 
                      vector<RealArray>& IonFrac);
\end{verbatim}

There are also handy routines to return and index into the
temperatures and the number of temperatures

\begin{verbatim}
int getNEItempIndex(const Real& tkeV);
int getNEInumbTemp();
\end{verbatim}

to providing debugging information

\begin{verbatim}
string writeIonFrac(const IntegerArray& Zarray, 
                    const vector<RealArray>& IonFrac);
string writeIonFrac(const int& Z, const IntegerArray& Zarray, 
		    const vector<RealArray>& IonFrac);
\end{verbatim}

to do a binary search on a RealArray and return the index of
the element immediately less than the input target

\begin{verbatim}
int locateIndex(const RealArray& xx, const Real x);
\end{verbatim}

and to check whether arrays are identical (note that the C++11
standard includes this as part of the valarray class but this is not
yet implemented in all compilers.

\begin{verbatim}
bool identicalArrays(const vector<RealArray>& a, const vector<RealArray>& b);
bool identicalArrays(const RealArray& a, const RealArray& b);
bool identicalArrays(const IntegerArray& a, const IntegerArray& b);
\end{verbatim}


\section{Examples}

A calcNEIfractions use can be found in the file 
vvgnei.cxx in the directory Xspec/src/XSFunctions.

\chapter{NeutralOpacity}

This class serves as a limited C++ interface to the Fortran gphoto and
photo routines to parallel the IonizedOpacity class. It will use the
cross-sections set using the xsect command. The method Setup should be
used to initialize the object then GetValue or Get to return opacities
for a single or multiple energies, respectively. IronAbundance is used
to set the iron abundance (relative to the defined Solar) and
Abundance the abundances of all other elements. IncludeHHe specifies
whether to include the contributions of hydrogen and helium in
the total opacity. 

\begin{verbatim}
class NeutralOpacity{
 public:

  IntegerArray AtomicNumber;
  vector<string> ElementName;

  string CrossSectionSource;

  NeutralOpacity();     // default constructor
  ~NeutralOpacity();    // destructor

  void Setup();   // set up opacities
  void Get(RealArray inputEnergy, Real Abundance, Real IronAbundance, 
           bool IncludeHHe, RealArray& Opacity);  // return opacities 
  void GetValue(Real inputEnergy, Real Abundance, Real IronAbundance, 
                bool IncludeHHe, Real& Opacity);  // return single opacity 

};
\end{verbatim}

\section{Examples}

An example use of NeutralOpacity can be found in the routine
calcCompReflTotalFlux in the file MZCompRefl.cxx in the directory 
Xspec/src/XSFunctions.

\chapter{IonizedOpacity}

This class calculates opacity of a photo-ionized material. At present
it uses the Reilman \& Manson (1979) opacities although this could be
generalized in future. The Setup method reads the input files if
necessary and calculate ion fractions for the requested ionization
parameter, temperature and spectrum. GetValue or Get then return opacities
for a single or multiple energies, respectively. IronAbundance is used
to set the iron abundance (relative to the defined Solar) and
Abundance the abundances of all other elements. IncludeHHe specifies
whether to include the contributions of hydrogen and helium in
the total opacity. 

\begin{verbatim}
class IonizedOpacity{
 public:

  IntegerArray AtomicNumber;
  vector<string> ElementName;

  RealArray Energy;

  RealArray **ion;
  RealArray **sigma;
  RealArray *num;

  IonizedOpacity();     // default constructor
  ~IonizedOpacity();    // destructor

  void LoadFiles();     // internal routine to load model data files
  void Setup(Real Xi, Real Temp, RealArray inputEnergy, 
             RealArray inputSpectrum);   // set up opacities
  void Get(RealArray inputEnergy, Real Abundance, Real IronAbundance, 
           bool IncludeHHe, RealArray& Opacity);  // return opacities 
  void GetValue(Real inputEnergy, Real Abundance, Real IronAbundance, 
                bool IncludeHHe, Real& Opacity);  // return single opacity 

};
\end{verbatim}

\section{Examples}

An example use of IonizedOpacity can be found in the routine
calcCompReflTotalFlux in the file MZCompRefl.cxx in the directory 
Xspec/src/XSFunctions.

\chapter{MZCompRefl}

This class calculates Compton reflection using the Magzdiarz and
Zdziarski code. 

\begin{verbatim}
class MZCompRefl{
 public:

  MZCompRefl();         // default constructor
  ~MZCompRefl();        // default destructor

  void CalcReflection(string RootName, Real cosIncl, Real xnor, 
                      Real Xmax, RealArray& InputX, RealArray& InputSpec, 
                      RealArray& Spref);
\end{verbatim}

The easiest way to use the MZCompRefl class is through the associated
function calcCompReflTotalFlux.

\begin{verbatim}
void calcCompReflTotalFlux(string ModelName, Real Scale, Real cosIncl, 
                           Real Abund, Real FeAbund, Real Xi, Real Temp, 
                           Real inXmax, RealArray& X, RealArray& Spinc, 
                           RealArray& Sptot);
\end{verbatim}

ModelName is the name of model and is used to check the
ModelName\_PRECISION xset variable to determine the precision to which
internal integrals should be calculated. 

Scale is the fraction of reflected emission to include. If Scale is
zero then no reflection component is included, a value of one
corresponds to an isotropic source above an infinite disk. The special
case of minus one will return only the reflected component.

cosIncl is the cosine of the inclination angle of the disk to the line
of sight. The iron abundance is specified by FeAbund and the
abundances of all other elements by Abund.

For an ionized disk Xi and Temp give the ionization parameter and
temperature for the IonizedOpacity class. If Xi is zero then the
NeutralOpacity class is used instead. In both these cases the boolean
IncludeHHe is set to false.

The input energies and spectrum are given by the X and Spinc
arrays. The energies are in units of $m_e c^2$ and the input spectrum
is $E F_E$. The output spectrum Sptot is also $E F_E$. The input
variable inXmax is the maximum value of X for which the reflected
spectrum is calculated. This is useful because X and Spinc should be
specified to a higher energy than required in the output because the
energy downscattering in Compton reflection. Setting inXmax to the
highest output energy required will save computation time.

\section{Examples}

An example use of calcCompReflTotalFlux can be found in the routine
doreflect in the file ireflct.cxx in the directory Xspec/src/XSFunctions.

\end{document}
