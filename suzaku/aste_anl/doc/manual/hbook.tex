\chapter{HBOOK}\label{chap:HBOOK}

To use HBOOK,
you have to know only two concepts at least,
{\em i.e.},
booking histograms and filling data to them.
Histograms can be one dimensional or two dimensional.
HBOOK supports another type of histograms
and various useful functions to operate them.
See {\em HBOOK Reference Manual} \footnote
{{\em HBOOK Reference Manual} is distributed with HBOOK by CERN.}
in details.

We hope following descriptions will helpful for you.

\begin{enumerate}

\item Book Histogram \\
--- One dimensional case ---
\begin{quote}
\baselineskip 3.2mm\begin{verbatim}
Call HBOOK1 (ID, TITLE, NX, XMIN, XMAX, 0.0)
\end{verbatim}
\begin{tabbing} mmmmmm \= mmmmmmmmmmmmmmmmmmm \= m \kill
Input parameter \\
{\tt ID}	\> Histogram Identifer		\> (INTEGER $>$ 0) \\
{\tt TITLE}	\> Histogram Title		\> (CHARACTER) \\
{\tt NX}	\> Number of channels		\> (INTEGER) \\
{\tt XMIN}	\> Lower edge of first channel	\> (REAL) \\
{\tt XMAX}	\> Upper edge of last channel	\> (REAL)
\end {tabbing}
\end{quote}

--- two dimensional case ---

\begin{quote}
\baselineskip 3.2mm\begin{verbatim}
Call HBOOK2 (ID, TITLE, NX, XMIN, XMAX, NY, YMIN, YMAX, 0.0)
\end{verbatim}
\begin{tabbing} mmmmmm \= mmmmmmmmmmmmmmmmmmm \= m \kill
Input parameter \\
{\tt ID}	\> Histogram Identifer			\> (INTEGER $>$ 0) \\
{\tt TITLE}	\> Histogram Title			\> (CHARACTER) \\
{\tt NX}	\> Number of channels in X		\> (INTEGER) \\
{\tt XMIN}	\> Lower edge of first X channel	\> (REAL) \\
{\tt XMAX}	\> Upper edge of last X channel		\> (REAL) \\
{\tt NY}	\> Number of channels in Y		\> (INTEGER) \\
{\tt YMIN}	\> Lower edge of first Y channel	\> (REAL) \\
{\tt YMAX}	\> Upper edge of last Y channel		\> (REAL)
\end {tabbing}
\end{quote}

\item Fill Data into Histogram \\
--- One dimensional case ---

\begin{quote}
\baselineskip 3.2mm\begin{verbatim}
Call HF1 (ID, X, WEIGHT)
\end{verbatim}
\begin{tabbing} mmmmmm \= mmmmmmmmmmmmmmmmmmm \= m \kill
Input parameter \\
{\tt ID}	\> Histogram Identifer		\> (INTEGER $>$ 0) \\
{\tt X}		\> value of the abscissa	\> (REAL) \\
{\tt WEIGHT}	\> event weight			\> (REAL)
\end {tabbing}
\end{quote}

--- two dimensional case ---

\begin{quote}
\baselineskip 3.2mm\begin{verbatim}
Call HF2 (ID, X, Y, WEIGHT)
\end{verbatim}
\begin{tabbing} mmmmmm \= mmmmmmmmmmmmmmmmmmm \= m \kill
Input parameter \\
{\tt ID}	\> Histogram Identifer		\> (INTEGER $>$ 0) \\
{\tt X}		\> value of the abscissa	\> (REAL) \\
{\tt Y}		\> value of the ordinate	\> (REAL) \\
{\tt WEIGHT}	\> event weight			\> (REAL)
\end {tabbing}
\end{quote}

\end{enumerate}


