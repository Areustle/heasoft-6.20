\chapter{Introduction}
The ASCA\_ANL is an
%"humanistic"
 analysis frame work
\footnote{The ASCA\_ANL is  "humanistic" software rather than
"animal" software.
The key of the ANL system is the "software-pipeline".
The concept of the ANL system was first invented by the
PEP4 group at LBL and the scheme was used in the
analysis extensively.}
for the ASCA experiment
specially designed for those
who may write programs for their own analysis in "human" way.
It accepts a set of user-supplied\footnote{
In the ASCA\_ANL world a "user" means
a person who write programs for their analysis.
Thus, you can say a user is a programmer and analyzer.
}
simple modules
and handles the execution sequence of the modules.
The sequence can be dynamically changed
and a user can specify the modules to be executed
and the order of execution.


The ASCA\_ANL provides standard environment for the programmer.
It has several built-in modules
 to access
photon information in the science FITS file
together withe the  orbital information in the orbit FRF and
the HK FITS file\footnote{
the BNK/EVS system is used to handle the communication
between modules.
}. This feature enables a user to analyze ASCA data in a "human" way.
It reads data from the FITS file produced by the FRFread/ASCALIN/XSELECT
and pass them to the user's module.
Thus the user does not have to know details of FITS I/O
nor a format of the orbit FRF.
Furthermore,
he/she is allowed to perform his/her original analysis
because the module can be written easily as the user pleases,
even if it is not supported in XSELECT.

By standardizing the way to write programs,
the user is benefited in several ways from the analysis point of view.
First,
since most part of user interface is provided by the ASCA\_ANL system,
all software written in the ASCA\_ANL looks alike.
The same look-and-feel is saves your time
to learn  how to use a new program every time.
Second,
the user can easily  combine modules written by different people.
Numerous ways of analysis can be realized
simply by combining "ready-made" modules.\footnote{
For example,
the ASCA\_ANL has a built-in module that reads FRF file directly.
Since the system manages to deliver the photon data is the same
way as the one for FITS data,
the same module can analyze data both from FITS files and from FRF files.
}
Third,
Once a user module is established in the ASCA\_ANL,
the module can be easily transformed to a standard FTOOL package\footnote{
By just adding standard I/O routines required for the FTOOL package.
}
as expected by people living in the "human" world.


The ASCA\_ANL uses various software packages
in high energy experiments developed so far.
Especially, HBOOK4\footnote{
A standard histogram  package in CERN Library.
}
is a standard histogram package in the system
that allows the user to make as many histograms (spectra) as he/she desires.  
DISPLAY45\footnote{
DISPLAY45 is a command-driven interactive histogram browsing program
} reads histogram files stored by HBOOK4 in the ASCA\_ANL
and allows the user to operate histograms in various ways.
These tools are useful and powerful
as well as they are well-maintained.
In addition,
it converts a format of the histogram into a spectral FITS format
and saves it in a spectral FITS file for XSPEC.
%This function helps you to shift smoothly into the "human" world
%without throwing away your software property.


In this documentation, we introduce a concept of ANL system.
We present a "walk-through" guide,
instructions to access data,
and a template of a module.
Also, an example of user modules are listed in appendix.

